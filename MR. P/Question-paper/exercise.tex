\documentclass[a4paper,10pt]{article}
\usepackage[left=1.5cm, right=1.5cm, top=2.5cm, bottom=2.5cm]{geometry}
\usepackage{fancyhdr}
\usepackage{amsmath}
\usepackage{enumerate}
\usepackage{tikz}
\usepackage{graphicx}
\usepackage{multicol}
\pagestyle{fancy}
\usepackage{fancyheadings}
%\fancyhead[C]{Aaegis}
\fancyfoot[C]{\textbf {Aaegis Academy, Opp. Atul Sweets, Jehan Circle, Gangapur
Road, Nasik}}
\lhead{Board Level Exercise}
\rhead{AAEGIS ACADEMY}
%\usepackage[usenames, dvipsnames]{color}
%===================================
\usepackage{draftwatermark}
\SetWatermarkText{\includegraphics{images/logo.png}}
\SetWatermarkLightness{0.1}
\SetWatermarkScale{1.0}
%============================
%\usepackage[printwatermark]{xwatermark}
%\savebox\mybox{\tikz[opacity=0.05]\node{
%\includegraphics{images/logo2.PNG}};} % MK path
%\newwatermark*[allpages,color=red!50,angle=45,scale=3,xpos=0,ypos=0]{images/logo2.PNG}
%\newwatermark*[

 %allpages,  angle=45,  scale=1,  xpos=0,  ypos=0] {\usebox\mybox}
%%================================ MACRO =================%%
\newcommand{\exercise}[1]{\begin{center}\hrule \vspace{5mm}\Huge
{\textbf{#1}} \vspace{5mm} \hrule \end{center}}
\newcommand{\questiontype}[1]{\begin{center}\hrule \vspace{2mm}
\Large{\textbf{#1}}\vspace{2mm}\hrule \end{center}}
\renewcommand\thesection{\Alph{section}}
\renewcommand{\thesubsection}{\thesection-\arabic{subsection}}
\setcounter{section}{0}
%%======================================
\begin{document}
\begin{center}
\Huge{\textbf {Board Level Exercise}}
\vspace{6mm}
\hrule
\end{center}
\vspace{5mm}
\begin{enumerate}[Type (I):]
\item Very Short Answer Type Questions:\hspace{50ex}{\textbf {01 Mark
Each}}
\begin{enumerate}[1.]
\item Write the principal value of $sec^{2}(-2)$.
\vspace{5mm}
\item If \(tan^{-1}\sqrt{3}+cot^{-1}(x)=\dfrac{\pi}{2},\)find $x$.
\end{enumerate}
\item Short Answer Type Questions : \hspace{55ex}{\textbf {02 Mark
Each}}

\begin{enumerate}[1.]
%\setcounter{enumi}{3}
\item If $sin^{-1}(x)+cos^{-1}\big(\dfrac{1}{2}\big)=\dfrac{\pi}{2}$. then find
$x$.
\vspace{5mm} 
\item Solve for $x$: $cos(2sin^1x)=\dfrac{1}{9}, x>0$.
\end{enumerate}
\item Long Answer Type Questions:\hspace{57ex}{\textbf {04 Mark
Each}}

\begin{enumerate}[1.]
\item Solve the following for $x$ :
$tan^1\left[\dfrac{1+x}{1-x}\right]=\dfrac{\pi}{4}+tan^1x, 0<x<1$.
\vspace{5mm}
\item Solve the $x$ :
$cos^{-1}x+sin^1\big(\dfrac{x}{2}\big)=\dfrac{\pi}{6}$. 
\vspace{5mm}
\item Prove the following :
$2tan^{-1}\dfrac{1}{3}+tan^{-1}\dfrac{1}{7}=\dfrac{\pi}{4}$.
\vspace{5mm}
\item Prove the following :
$cos\Bigg(sin^{-1}\dfrac{3}{5}+cot^{-1}\dfrac{3}{2}\Bigg)=\dfrac{6}{5
\sqrt{13}}$.
\end{enumerate}
\item Very Long Answer Type Questions :\hspace{51ex}{\textbf {06 Mark
Each}}

\begin{enumerate}[1.]
\item if $tan^{-1}x+tan^{-1}y+tan^{-1}z=\pi,$ prove that $x+y+z=xyz$.
\vspace{5mm}
\item Prove that :
$sin^{-1}\bigg(\dfrac{4}{5}\bigg)+sin^{-1}\bigg(\dfrac{5}{13}\bigg)+sin^{-1}\bigg(\dfrac{16}{65}\bigg)=\dfrac{\pi}{2}$.
\vspace{5mm}
\item Solve the following for $x$ : $tan^{-1}x+2 cot^{-1}x=\dfrac{2\pi}{3}$
\vspace{5mm}
\item Solve for $x$ :
$tan^{-1}\dfrac{x}{2}+tan^{-1}\dfrac{x}{3}=\dfrac{\pi}{4};\sqrt{6}>x>0$.
\vspace{5mm}
\item Prove that :
$tan^{-1}\dfrac{1}{4}+tan^{-1}\dfrac{2}{9}=\dfrac{1}{2}tan^{-1}\dfrac{4}{3}$.
\vspace{5mm}
\item Prove that :
$2tan^{-1}\dfrac{3}{4}-tan^{-1}\dfrac{17}{31}=\dfrac{\pi}{4}$.
\vspace{5mm} 
\item Solve for $x$ :
$tan^{-1}\bigg(\dfrac{2x}{1-x^2}\bigg)+cot^{-1}\bigg(\dfrac{1-x^2}{2x}\bigg)=\dfrac{\pi}{3},-1<x<1$.
\vspace{5mm} 
\item Solve for $x$
:$tan^{-1}(x+2)+tan^{-1}(x-2)=tan^{-1}\bigg(\dfrac{8}{79}\bigg) ; x>0$.
\vspace{5mm} 
\item Prove that : $tan^{-1}(1)+tan^{-1}(2)+tan^{-1}(3)=\pi$.
\vspace{5mm} 
\item Prove the following :
$tan^{-1}\bigg(\dfrac{3}{4}\bigg)+tan^{-1}\bigg(\dfrac{3}{5}\bigg)-tan^{-1}\bigg(\dfrac{8}{19}\bigg)=\dfrac{\pi}{4}$.
\end{enumerate}
\end{enumerate}
%============ End=================%
\exercise{Exercise \# 1}
\questiontype{PART - I : SUBJECTIVE QUESTIONS}
\section{Definition, graphs and fundamentals}
\subsection{Find the simplified value of each of the following inverse
trigonometric terms :}
\vspace{5mm} 
\begin{multicols}{2}
\begin{enumerate}[(i)]
\item $sin^{-1}\bigg(\dfrac{1}{2}\bigg)$
\item $cos^{-1}\bigg(\dfrac{\sqrt{3}}{2}\bigg)$
\item $cosec^{-1}\bigg(-\dfrac{1}{2}\bigg)$
\item $cos^{-1}\bigg(-\dfrac{1}{2}\bigg)$
\item $sec^{-1}(-\sqrt{2})$
\end{enumerate}
\end{multicols}
\subsection{Find the simplified value of the following expressions :}
\begin{multicols}{2}
\begin{enumerate}[(i)]
\item $sin\bigg[\dfrac{\pi}{3}-sin^{-1}\bigg(-\dfrac{1}{2}\bigg)\bigg]$
\item
$tan\bigg[cos^{-1}\dfrac{1}{2}+tan^{-1}\bigg(-\dfrac{1}{\sqrt{3}}\bigg)\bigg]$
\item
$sin^{-1}\bigg[\bigg\{sin^{-1}\bigg(\dfrac{\sqrt{3}}{2}\bigg)\bigg\}cos\bigg]$
\end{enumerate}
\end{multicols}
\subsection{Draw the graph of the following functions :}
\begin{multicols}{2}
\begin{enumerate}[(i)]
\item $y = sin^{-1}(x+1)$
\item $y = cos^{-1}(3x)$
\item $y = tan^{-1}(2x-1)$
\end{enumerate}
\end{multicols}
\subsection{Solve the following inequalities :}
\begin{multicols}{2}
\begin{enumerate}[(i)]
\item $sin^{-1} x > - 1$
\item $cos^{-1} x < 2$
\item $cot^{-1} x < -\sqrt{3}$
\end{enumerate}
\end{multicols}
\subsection{}
\begin{enumerate}[(i)]
\vspace{3mm}
\item If $\sum_{i=1}^{n} cos^{-1} \alpha_i = 0,$ then find the value of
$\sum_{i=1}^{n} i \alpha_i$
\vspace{5mm}
\item If $\sum_{i=1}^{2n}sin^{-1} x_i = n\pi,$ then show that
$\sum_{i=1}^{2n} x_i = 2n$
\end{enumerate}
\section{Trig. (trig -1 $x$), trig -1 (trig $x$) trig (-$x$)}
\subsection{Evaluate the following expressions :}
\begin{multicols}{2}
\begin{enumerate}[(i)]
\item $sin\bigg(cos^{-1}\dfrac{3}{5}\bigg)$
\item $tan\bigg(cos^{-1}\dfrac{1}{3}\bigg)$
\item $cosec\bigg(sec^{-1}\dfrac{\sqrt{41}}{5}\bigg)$
\item $tan\bigg(cosec^{-1}\dfrac{65}{63}\bigg)$
\item
$sec\bigg(tan\bigg\{tan^{-1}\bigg(-\dfrac{\pi}{3}\bigg)\bigg\}\bigg)$
$cos tan^{-1} {sin} cot^{-1}\dfrac{1}{2}$
\end{enumerate}
\end{multicols}
\subsection{Evaluate the following inverse trigonometric expressions :}
\begin{multicols}{2}  
\begin{enumerate}[(i)] 
\item $sin^{-1}\bigg(sin\dfrac{7\pi}{6}\bigg)$
\item $tan^{-1}\bigg(tan\dfrac{2\pi}{3}\bigg)$
\item $cos^{-1}\bigg(cos\dfrac{5\pi}{4}\bigg)$
\item $sec^{-1}\bigg(sec\dfrac{7\pi}{4}\bigg)$
\end{enumerate}
\end{multicols}
\subsection{Find the value of the following inverse trigonometric
expressions :}
\begin{multicols}{2}     
\begin{enumerate}[(i)] 
\item $sin^{-1}(sin 4)$
\item $cos^{-1}(cos 10)$
\item $tan^{-1}(tan (-6))$
\item $cot^{-1}(cot(-10))$
\item
$cos^-1\bigg(\dfrac{1}{\sqrt{2}}\bigg(cos\dfrac{9\pi}{10}-sin{9\pi}{10}\bigg)\bigg)$
\end{enumerate}
\end{multicols}
\subsection{}
Express $sin^{–1} (sin \theta, cos –1 (cos \theta, tan –1 (tan \theta
\text{ and } cot –1 (cot
\theta$ in terms of linear expression of $\theta$ for $\theta
\epsilon\bigg[\dfrac{3\pi}{2},3\pi\bigg]$
\section{Property "$\dfrac{\pi}{2}$" , Addition and subtraction rule,
miscellaneous formula, summation of series}
\subsection{Find the value of following expressions :}
\begin{multicols}{2}     
\begin{enumerate}[(i)] 
\item $cot (tan^{-1} a+cot^{-1} a)$
\item $sin(sin^{-1}x+cos^{-1}x), |x| \leq 1$
\item
$tan\bigg[cos^{-1}\bigg(\dfrac{3}{4}\bigg)+sin^{-1}\bigg(\dfrac{3}{4}\bigg)-sec^{-1}3\bigg]$
\end{enumerate}
\end{multicols}
\subsection{Prove that}
\begin{enumerate}[(i)]  
\item
$sin^{-1}\bigg(\dfrac{3}{5}\bigg)+sin^{-1}\bigg(\dfrac{8}{17}\bigg) =
sin^{-1}\dfrac{77}{85}$
\item $cos^{-1}\dfrac{4}{5}+cos^{-1}{12}{13} = cos^{-1}\dfrac{33}{65}$
\item $sin^{-1}\bigg(\dfrac{1}{\sqrt{5}}\bigg)+cot^{-1}3 =
\dfrac{\pi}{4}$
\item
$tan^{-1}\bigg(\dfrac{1}{3}\bigg)+tan^{-1}\bigg(\dfrac{1}{5}\bigg)+tan^{-1}\bigg(\dfrac{1}{7}\bigg)+tan^{-1}\bigg(\dfrac{1}{8}\bigg)
= \dfrac{\pi}{4}$
\end{enumerate}
\subsection{Simplify
$tan^{-1}\bigg\{\dfrac{1}{2}sin^{-1}\bigg(\dfrac{2x}{1+x^2}\bigg)+\dfrac{1}{2}cos^{-1}\bigg(\dfrac{1-y^2}{1+y^2}\bigg)\bigg\}$,
if $x > y > 1.$} 
\vspace{4mm}
\subsection{Find the value of
$sin^{-1}(cos(sin^{-1}x))+cos^{-1}(sin(cos^{-1}x))$}
\section{Inverse trigonometric function Equations :}
\subsection{Solve for $x$}
\begin{multicols}{2}
\begin{enumerate}[(i)]
\item $cos(2sin^{-1}x = \dfrac{1}{3}$
\item $cot^{-1}x+tan^{-1}3 = \dfrac{\pi}{2}$
\end{enumerate}
\end{multicols}
\subsection{Solve the following equations :}
\begin{multicols}{2}
\begin{enumerate}[(i)]
\item
$tan^{-1}\bigg(\dfrac{x-1}{x-2}\bigg)+\dfrac{1}{2}tan^{-1}\bigg(\dfrac{x+1}{x+2}\bigg)
= \dfrac{\pi}{4}$
\item $sin^{-1}x+sin^{-1}2x = \dfrac{2\pi}{3}$  
\end{enumerate}
\end{multicols}
\subsection{Solve the following equations :}
\begin{enumerate}[(i)]
\item $tan^{-1}\bigg(\dfrac{1-x}{1+x}\bigg) = \dfrac{1}{2}tan^{-1}x,
(x>0)$
\item
$3tan^{-1}\bigg(\dfrac{1}{2+\sqrt{3}}\bigg)-tan^{-1}\bigg(\dfrac{1}{x}\bigg)
= tan^{-1}\bigg(\dfrac{1}{3}\bigg)$
\end{enumerate}
\usecounter{section}
%\section{Objective q}
\end{document}
