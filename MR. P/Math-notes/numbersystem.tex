\chapter{Number System}
%\questiontype{Chapter 1 : Number System}
\section{Classification of Numbers}
Undoubtedly numbers are the cornerstone of mathematics. There is a huge
literature on the
development of numbers but that is beyond our scope. We directly jump to
the classification
of numbers. Remember this classification looks so obvious or trivial
today but humans took centuries to reach here. Reader should appreciate
this.\\
%%%%%%%%%%%%%% CLASSIFICATION TREE %%%%%%%%%%%%%%%%%%%%%%%%
\begin{center}
\begin{forest}
  forked edges,
  for tree={edge+={-Latex}},
  [Numbers
    [Real Numbers
      [Rational Numbers
        [Fractions]
        [Integers
        [Negative Integers]
        [Non-negative Integers (Whole Numbers)
        [$\big\{\hspace{0.1mm}0\big\}$]
        [Positive Integers
(Natural numbers)]
        ]
        ]
      ]
      [Irrational Numbers]
    ]
    [Imaginary Number]
  ]
\end{forest}
\end{center}
%%%%%%%%%%%%%%%%%%%%%%%%%%%%%%%%%%%%%%%%%%%%%%%%%%%%%%%%%
 Here are some notations to denote the set of numbers :\\\\
 $\mathbb{N}$ - Set of Natural Numbers = $\big\{1,2,3,....\big\}$\\
 $\mathbb{N}_0$ or $\mathbb{W}$ - Set of whole numbers = $\big\{0,1,2,...\big\}$\\
$\mathbb{Z}$- Set of integers = $\big\{....-5,-4,...-1,0,1,2...\big\}$\\
$\mathbb{Q}$ - Set of rational numbers =
$\big\{...-5,\dfrac{-4}{7},\dfrac{-1}{2},0,\dfrac{1}{2},\dfrac{1}{3},\dfrac{5}{7},4,...\big\}$\\\\
%* All integers are rational numbers\\\\
$\mathbb{Q'}$ or $\mathbb{Q}^c$ - Set of irrational Numbers =
$\big\{...-\sqrt{11},-\sqrt{5},-\sqrt{7},-\dfrac{1}{\sqrt{3}}....\sqrt{5},...\big\}$\\
$\mathbb{R}$ - Set of real Number = $\big\{...-\sqrt{11},-\sqrt{\dfrac{5}{7}},-1.5,-1,0,\sqrt{2},\dfrac{1}{5}...\big\}$\\
A number is said to be rational if it can be written as $\dfrac{p}{q}$ form where
p, q are integers. For example
$\dfrac{1}{2},\dfrac{-1}{5},\dfrac{1}{-5},-\dfrac{2}{7},-1,0$,
$4,-\dfrac{7}{8}$
are rational numbers. A real number is said to be irrational if it is
not rational. For example
$\sqrt{2},\sqrt{3},\sqrt{5},-\sqrt{2},\dfrac{1}{\sqrt{2}}$,
$-\dfrac{\sqrt{3}}{\sqrt{2}}$
are irrational number. (It is very interesting to prove that
$\sqrt{2},\sqrt{3}$ etc. cannot be written as $\dfrac{p}{q}$ form. We
shall prove this shortly). 
\subsection{Exercises}
Q.1 Prove that $\dfrac{a+b}{2}$ lies between $a$ and $b (a < b).$\\
Q.2 {Find five rational numbers between.}
\begin{multicols}{5}
\begin{enumerate}[(i)]
\item 1 and 10
\item 11 and 12
\item $\dfrac{1}{3}$ and $\dfrac{1}{2}$
\item $-\dfrac{1}{5}$ and $-\dfrac{1}{7}$
\item $-\dfrac{3}{5}$ and $\dfrac{3}{5}$  
\end{enumerate}
\end{multicols}
\noindent Q.3 {State true or false}
\begin{enumerate}[(i)]
\item Every rational number is an integer.
\item Every natural number is an integer.
\item Every integer is an rational number.
\item 5 is an non-negative number.
\item 0 is an non-negative integer.
\item Every irrational number is an rational number.
\item $-\dfrac{1}{7}$ and 7 are rational numbers.
\end{enumerate}
Q.4 {Express the following in decimal by long division}
\begin{multicols}{5}
\begin{enumerate}[(i)]
\item $\dfrac{19}{5}$
\item $\dfrac{3}{15}$
\item $-\dfrac{19}{2}$
\item $\dfrac{2157}{625}$
\item $-\dfrac{17}{8}$
\item $\dfrac{327}{500}$
\end{enumerate}
\end{multicols}
\noindent Q.5 {Show that the decimal expansion of following rational
numbers are non-terminating and
repeating.}
\begin{multicols}{4}
\begin{enumerate}[(i)]
 \item $\dfrac{8}{3}$
 \item $\dfrac{2}{11}$
 \item $\dfrac{1}{7}$
 \item $-\dfrac{16}{45}$
\end{enumerate}
\end{multicols}
%=========== End section 1==============%
\section{Non-terminating and recurring rational number}
There are rational numbers such that when we try to express them in
decimal form by
division method, we find that no matter how long we divide there is
always a remainder. In
other words, the division process never comes to an end. This is due to
the reason that in the
division process the remainder starts repeating after a certain number
of steps. In such cases,
a digit of a block of digits repeats itself. For example,
0.3333...,0.1666666...,
0.123123123...., 1.2692307692307692307...etc. Such decimals are called
non-terminating
repeating decimals or non-terminating recurring decimals. These decimal
numbers are
represented by putting a bar over the first block of the repeating part
and omit the other
repeating blocks. Thus, we write 0.33333... = $0.\overline{3}$, 0.16666... =
$0.1\overline{6}$, 0.123123123 ..... = $0.\overline{123}$ and
1.26923076923076292307... = $1.2\overline{692307}$.\\\\
\fbox{Fact : Every non terminating and recurring number is a rational
number.}\\\\
Q.1 Express the following rational numbers as decimals
\begin{multicols}{5}
\begin{enumerate}[(i)]
 \item $\dfrac{2}{3}$
 \item $-\dfrac{4}{5}$
 \item $-\dfrac{2}{15}$
 \item $-\dfrac{22}{13}$
 \item $\dfrac{437}{999}$
 \item $\dfrac{33}{26}$
\end{enumerate}
\end{multicols}
\noindent Q.2 Look at several examples of rational numbers in the form
$\dfrac{p}{q} (q \neq 0)$, where p and q are integers with no common
factors other than 1 and having terminating decimal representations. Can
you guess what property q must satisfy?
%===================== SECTION 2 END======================%
\section{Conversion of non-terminating recurring rational numbers in
the form $\dfrac{p}{q}$}
Q.1 Express $0.\overline{4}$ in the form $\dfrac{p}{q}$\\
Q.2 Express each of the following decimals in the form $\dfrac{p}{q}$
\begin{multicols}{2}
\begin{enumerate}[(i)]
\item $0.\overline{35}$
\item $0.\overline{585}$
\end{enumerate}
\end{multicols}
\noindent Q.3 Express the following decimals in the form $\dfrac{p}{q}$
\begin{multicols}{3}
\begin{enumerate}[(i)]
\item $0.3\overline{2}$
\item $0.12\overline{3}$
\item $0.003\overline{52}$
\end{enumerate}
\end{multicols}
\noindent Q.4 Express each of the following decimals in the form $\dfrac{p}{q}$
\begin{multicols}{7}
\begin{enumerate}[(i)]
\item $0.\overline{4}$
\item $0.\overline{37}$
\item $0.\overline{54}$
\item $0.\overline{621}$
\item $125.\overline{3}$
\item $4.\overline{7}$
\item $0.4\overline{7}$
\end{enumerate}
\end{multicols}
\vspace{10mm}
%%%%%%%%%%%%%%%%%%%%%%%% SOLUTION %%%%%%%%%%%%%%%%%%%%%%
\noindent Q.1 Solution :\\\\
Let $x = 0.\overline{4} = 0.444$\\
$10x = 4.444....$\\
$10x-x = 4$\\
$9x  = 4 = x = \dfrac{4}{9}$\\\\
Q.2 Solution :\\\\
(i) Let $x = 0.\overline{35}$\\
$x = 0.353535......(i)$\\
Here, we have two repeating digits after the decimal point. So, we
multiply sides of (i) by $10^2 = 100$ to get\\
100 $x = 35.3535$... (ii)\\
Subtracting (i) from (ii), we get\\
$100x-x = (35.3535...) – (0.3535...)$\\
$99x = 35$\\
$X = \dfrac{35}{99}$\\
Hence, $0.\overline{35} = \dfrac{35}{99}$\\\\
%%%%%%%%%%%%%%%%%%%
(ii) Let $x = 0.\overline{585}$\\
$x = 0.585585585...$\\
Here, we have three repeating digits after the decimal point. So, we
multiply both sides of (i) by $10^3 =  1000$ to get\\
1000 $x = 585.585585...$\\
Subtracting (i) from (ii), we get\\
$1000 x-x = (585.585585...)- (0.585585585...)$\\
$1000 x-x = 585$\\
$999x = 585$\\
$ x = \dfrac{585}{999}$\\\\
Q.3 Solution :\\\\
(i) Let $x = 0.3\overline{2}$ \\
Clearly, there is just one digit on the right side of the decimal point
which is without bar. So,
we multiply both sides of $x$ by 10 so that only the repeating decimal is
left on the right side of
the decimal point.\\
 i.e $10 x = 3.\overline{2}$\\
$\Rightarrow 10x = 3+0.\overline{2}$\\
$\Rightarrow 10x = 3+\dfrac{2}{9}$\\
$\Rightarrow 10x = \dfrac{9\times3+2}{9} \Rightarrow 10x = \dfrac{29}{9}
\Rightarrow x = \dfrac{29}{90}$\\\\
(ii) Let $x = 0.12\overline{3}$\\
Clearly, there are two digits on the righ side of the decimal point
which are without bar. So,
we multiply both sides of $x$ by 102 = 100 so that only the repeating
decimal is left on the right
side of the decimal point.\\
i.e $100x = 12.\overline{3}$\\
$\Rightarrow 100x = 12+0.\overline{3}$\\  
$\Rightarrow 100x = 12+\dfrac{3}{9}$\\
$\Rightarrow 100x = \dfrac{12\times9+3}{9}$\\
$\Rightarrow 100x = \dfrac{108+3}{9} \Rightarrow 100x = \dfrac{111}{9}
\Rightarrow x = \dfrac{111}{900}$\\\\
(iii) Let $x = 0.003\overline{52}$\\
Clearly, there are three digits on the right side of the decimal point
which are without bar. So,
we multiply both sides of $x$ by 103 = 1000 so that only the repeating
decimal is left on the
right side of the decimal point.\\
i.e $1000x = 3.\overline{52}$\\
$\Rightarrow  1000x =  3+0.52$\\
$\Rightarrow  1000x = 3+\dfrac{52}{99}$\\
$\Rightarrow  1000x = \dfrac{3\times99+52}{99}$\\
$\Rightarrow  1000x = \dfrac{297+52}{99} \Rightarrow  1000x =
\dfrac{349}{99} \Rightarrow  x = \dfrac{340}{99000}$\\\\
Q.4 Solution :
\begin{multicols}{6}
\begin{enumerate}[(i)]
\item $\dfrac{4}{9}$
\item  $\dfrac{37}{99}$
\item $\dfrac{6}{11}$
\item $\dfrac{23}{37}$
\item $\dfrac{376}{3}$
\item $\dfrac{43}{9}$
\item $\dfrac{43}{90}$
\end{enumerate}
\end{multicols}
%==============================
\begin{tcolorbox}
\begin{center}
\textbf{\Large Tricky Tip}\\
%\hrule
\vspace{3mm}
$\dfrac{p}{q} \text{  form} = \dfrac{\text{Complete Number - Number formed by non -
repeating digits}}{\text{No.of 9 as no.of repeating digits after that write
no.of 0 as no.of non - repeating digits}}$
\begin{enumerate}[(i)]
\item  $0.\overline{57} = \dfrac{57-0}{99} = \dfrac{57}{99}$
\item $0.\overline{347} = \dfrac{347-3}{990} = \dfrac{349}{990}$
\item $0.53\overline{763} = \dfrac{53763-53}{99900} =
\dfrac{53710}{99900} = \dfrac{5371}{9990}$
\end{enumerate}
%\hrule
\end{center}
\end{tcolorbox}
%%%%%%%%%%%%%%%%%%% END SECTION 3 %%%%%%%%%%%%%%%%%%%%%%%%%%
\section{Irrational Numbers (finding the square root by
division method)}
A number which is not rational is known as a irrational number. In other
words, a number which can’t be written as $\dfrac{p}{q}$ form.\\
\fbox{Fact : Every non-terminating and non-recurring number is irrational
number.}\\\\
$\sqrt{2},\sqrt{3},\sqrt{7},\pi$ are rational number.
Let’s find the value of $\sqrt{2},\sqrt{3}$ by division method.\\\\
1.\\
%%%%%%%%%%%%%%%%%%%%%%%%% DIVISION METHOD SUM 1 %%%%%%%%%%%%%%%
\begin{center}
\begin{tabular}{l|l } 
 
       & 1.4142135... \\\cline{2-2}
1      & 2.00000000000000 \\
       & 1 \\\cline{2-2} 
24     & 100\\ 
       &{\hspace{2mm}96}\\\cline{2-2} 
28     &{\hspace{4mm}400}\\
       &{\hspace{4mm}281}\\\cline{2-2}
2824   &{\hspace{4mm}11900}\\
       &{\hspace{4mm}11296}\\\cline{2-2} 
28282  &{\hspace{8mm}60400}\\ 
       &{\hspace{8mm}56564}\\\cline{2-2} 
282841 &{\hspace{10mm}383600}\\ 
       &{\hspace{10mm}282841}\\\cline{2-2}
2828423&{\hspace{10mm}10075900}\\ 
       &{\hspace{10mm}8485269}\\\cline{2-2}
28284265&{\hspace{10mm}159063100}\\
       &{\hspace{10mm}141421325}\\\cline{2-2} 
28284270&{\hspace{12mm}17611775}                              
\end{tabular}
\end{center}
%%%%%%%%%%%%%%%%%%%%%%%%%%%%%%%%%%%%%%%%%%%%%%%%%%%%%%%%%%
$\Rightarrow \sqrt{2} = 1.4142135...$\\\\
2.\\
%%%%%%%%%%%%%%%%%%%% DIVISION METHOD SUM 2 %%%%%%%%%%%%%%%
\begin{center}
\begin{tabular}{l|l } 
 
       & 1.732050807...\\\cline{2-2}
1      &3.00 00 00 00 00 00 00 00 \\
       & 1 \\\cline{2-2} 
27     & 200\\ 
       &{\hspace{2mm}189}\\\cline{2-2} 
343     &{\hspace{4mm}1100}\\
       &{\hspace{4mm}1029}\\\cline{2-2}
3462   &{\hspace{8mm}7100}\\
       &{\hspace{8mm}6924}\\\cline{2-2} 
346405  &{\hspace{8mm}1760000}\\ 
       &{\hspace{8mm}1732025}\\\cline{2-2} 
34641008 &{\hspace{12mm}279750000}\\ 
       &{\hspace{12mm}277128064}\\\cline{2-2}
3464101607&{\hspace{16mm}26219360000}\\ 
       &{\hspace{16mm}24248711249}\\\cline{2-2}
       &{\hspace{18mm}1970648751}\\
                                   
\end{tabular}
\end{center}
%%%%%%%%%%%%%%%%%%%%%%%%%%%%%%%%%%%%%%%%%%%%%%%%%%%%%%
$\Rightarrow \sqrt{3} = 1.732050807...$
\subsection{Exercises}
Q.1 Find the square root of following.
\begin{multicols}{5}
\begin{enumerate}[(i)]
\item 271441
\item 8281
\item 522729
\item 801025
\item 64009
\item 4664.89
\item 454.1161
\item 7
\item 8
\item 11
\end{enumerate}
\end{multicols}
%%%%%%%%%%%%%%%%%%%%%%%%%%%% End section 4 %%%%%%%%%%%%%%%%%%%
\section{Proof by Contradiction (Irrationality of $\sqrt{2}$)}
We will discuss this topic more thoroughly in the chapter ‘Mathematical
Logic’. Currently
our main motive is to prove the irrationality of $\sqrt{2} \text{ i.e } \sqrt{2}$
can not be written as $\dfrac{p}{q}$ form. This
method ‘proof by contradiction’ is used frequently in calculus, one of
the most interesting
branch of mathematics. In this method we assume the negation of
something which is to beproved then use deduce a contradiction which
implies that our assumption was wrong. Now let’s prove that $\sqrt{2}$ is an
irrational number.\\\\
Assume that $\sqrt{2}$ is an rational number it means it can be written
as $\dfrac{p}{q}$
form. Let\\
$\sqrt{2} = \dfrac{p}{q}$\\
where $p, q$ are co-prime ($p, q$ have not common factor) Now we have\\
$2 = \dfrac{p^2}{q^2}$\\
$p^2 = 2 q^2$\\\\
It shows that $p^2$ is an even integer, since $p$ is an integer it must be
an even integer. So take $p = 2m$ for some integer m. now we
have\\\\
$4 m^2 = 2q^2$\\
$2 m^2 = q^2$\\\\
It shows $q^2$ is an even integer hence $q$ is an even integer which
contradicts our assumption
that $p, q$ are co-prime. Hence our assumption was wrong that
$\sqrt{2}$ is an irrational number.
$\sqrt{2}$ is an rational number.\\\\ 
\subsection{Proposition}
For some $a,b,c,d \in \mathbb{Q}$ if $a+b\sqrt{2} = c+d\sqrt{2}$ then $a = c \text{ and } b = d.$(This result can be generalized if we replace $\sqrt{2}$ by any other irrational numbers).\\\\
Proof : We are given
\begin{equation}
\label{equ1}
    a+b\sqrt{2} = c+d\sqrt{2}
\end{equation}    
$$\Rightarrow (a-c) = (d-b)\sqrt{2}$$
Since $a,b,c,d \in Q$ then $(a-c) \in \mathbb{Q} \text{ and } (d-b) \in \mathbb{Q} \text{ but } (d-b)\sqrt{2} \in \mathbb{Q'}.$ So, L.H.S is an rational number and R.H.S is an irrational number. So, equality in (\ref{equ1}) can be hold only if $a-c = 0 \text{ and } d-b = 0$. So, $a = c \text{ and } b = d$

%%%%%%%%%%%%%%%%%%%%%%%%%%%%%%%%%%%%%%%%%%%%%%%%%%%%%
\subsection{Exercises}
Q.1 Prove that $\sqrt{3}$ is an irrational number\\
Q.2 Prove that $\sqrt{5}$ is an irrational number\\
Q.3 Prove that $\sqrt{n}$ is an irrational number if n is not a
perfect square.\\
Q.4 Prove that $\sqrt{3}+\sqrt{5}$ is an irrational number.\\
Q.5 Using the fact that negative of rational number is also a
rational number prove that
negative of irrational number is an irrational number.\\
Q.6 Prove that the sum of rational number and irrational
number is an irrational number.\\
Q.7 Prove a disprove that sum of two irrational number is an
irrational number.\\
Q.8 Prove that $\sqrt{ab}$ lies between two positive numbers between a
and b $(a < b)$. Hence, find two
irrational numbers between\\
\begin{multicols}{3}
\begin{enumerate}[(i)]
\item 2 and 3
\item $\sqrt{2}$ and $\sqrt{3}$
\item 2 and $\sqrt{5}$
\end{enumerate}
\end{multicols}
\noindent Q.9 Find $a,b \in \mathbb{Q}$ if $a+b\sqrt{5} = 3+7\sqrt{5}$.\\
Q.10 Find $x,y \in \mathbb{Q}$ if $(x+y)+(x-y)\sqrt{7} = 7+5\sqrt{7}.$
%%%%%%%%%%%%%%%%%%%%%%%%%%%%% END SECTION 5 %%%%%%%%%%%%%%%%%%%%
\section{Rationalization of numerator/denominator and comparison of surds}

\subsection{Example}
Rationalize the denominator
\begin{multicols}{3}
\begin{enumerate}[(i)]
\item $\dfrac{1}{3+\sqrt{2}}$
\item $\dfrac{7+\sqrt{5}}{7-\sqrt{5}}$
\item $\dfrac{3-\sqrt{7}}{2-\sqrt{7}}$
\end{enumerate}
\end{multicols}
\noindent Solution : 
(i) $\dfrac{1}{3+\sqrt{2}} = \dfrac{1}{3+\sqrt{2}}.\dfrac{(3-\sqrt{2)}}{(3-\sqrt{2})} = \dfrac{3-\sqrt{2}}{9-2} = \dfrac{3-\sqrt{2}}{5}$\\\\\\
(ii) $\dfrac{7+\sqrt{5}}{7-\sqrt{5}} = \dfrac{7+\sqrt{5}}{7-\sqrt{5}}.\dfrac{7+\sqrt{5}}{7+\sqrt{5}} = \dfrac{49+5+19\sqrt{5}}{44} = \dfrac{54+14\sqrt{5}}{44}
=\dfrac{54}{44}+\dfrac{14\sqrt{5}}{44} = \dfrac{27}{22}+\dfrac{7\sqrt{5}}{22}$\\\\\\
(iii) $\dfrac{3-\sqrt{7}}{2-\sqrt{7}} = \dfrac{3-\sqrt{7}}{3-\sqrt{7}}.\dfrac{2+\sqrt{7}}{2+\sqrt{7}} = \dfrac{6+3\sqrt{7}-2\sqrt{7+7}}{4-7}
= \dfrac{13+\sqrt{7}}{-3} = -\dfrac{13}{3}-\dfrac{1\sqrt{7}}{3}$
\subsection{Example}
Arrange $\sqrt[4]{6}, \sqrt[3]{7}, \sqrt{5}$ in ascending order.\\\\
Solution : 
$\sqrt[4]{6} = (6)^{\frac{1}{4}}, \sqrt[3]{7} = (7)^{\frac{1}{3}}, \sqrt{5} = (5)^{\frac{1}{2}}$\\
To determine the order of the surds we first make their exponents equal.\\
L.C.M of denominator of exponents 4,3,2 = 12\\
So, we have\\\\
$ 6^{\frac{1}{4}} = 6^{\frac{3}{12}} = (6^3)^{\frac{1}{2}} = (216)^{\frac{1}{12}}$\\\\
$7^{\frac{1}{3}} = 7^{\frac{4}{12}} = (7^{4})^{\frac{1}{12}} = (2401)^{\frac{1}{12}}$\\\\
$5^{\frac{1}{2}} = 5^{\frac{6}{12}} = (5^6)^{\frac{1}{12}} = (15625)^{\frac{1}{12}}$\\\\
So, $(216)^{\frac{1}{12}} < (2401)^\frac{1}{12} < (15625)^\frac{1}{12} $

$6^\frac{1}{4} < 7^\frac{1}{3} < 5^\frac{1}{2}$
\subsection{Exercises}
Q.1 Rationalize the denominator 
\begin{multicols}{3}
\begin{enumerate}[(i)]
\item $\dfrac{2}{\sqrt{7}}$
\item $\dfrac{2}{3\sqrt{3}}$
\item $\dfrac{5}{3-\sqrt{5}}$
\item $\dfrac{5+\sqrt{6}}{5-\sqrt{6}}$
\item $\dfrac{7+3\sqrt{5}}{7-3\sqrt{5}}$
\item $\dfrac{2\sqrt{3-\sqrt{5}}}{2\sqrt{2}+3\sqrt{3}}$
\end{enumerate}
\end{multicols}
\noindent Q.2 If $a,b$ are rational number, find $a$ and $b$.
\begin{multicols}{3}
\begin{enumerate}[(i)]
\item $\dfrac{\sqrt{3}-1}{\sqrt{3}+1} = a+b\sqrt{3}$
\item $\dfrac{3+\sqrt{7}}{3-\sqrt{7}} = a+b\sqrt{7}$
\item $\dfrac{5+2\sqrt{3}}{7+4\sqrt{3}} = a+b\sqrt{3}$
\item $\dfrac{5+\sqrt{3}}{7-4\sqrt{3}} = 47a+\sqrt{3b}$
\item $\dfrac{\sqrt{5}+\sqrt{3}}{\sqrt{5}-\sqrt{3}} = a+b\sqrt{15}$
\item $\dfrac{\sqrt{2}+\sqrt{3}}{3\sqrt{2}-2\sqrt{3}} = a-b\sqrt{6}$
\end{enumerate}
\end{multicols}
\noindent Q.3 Prove that
\begin{enumerate}[(i)]
\item $\dfrac{1}{3-\sqrt{8}} - \dfrac{1}{\sqrt{8}-\sqrt{7}} + \dfrac{1}{\sqrt{7}-\sqrt{6}} - \dfrac{1}{\sqrt{6}-\sqrt{5}} + \dfrac{1}{\sqrt{5}-2} = 5$
\item $\dfrac{1}{1+\sqrt{2}} + \dfrac{1}{\sqrt{2}+\sqrt{3}} + \dfrac{1}{\sqrt{3}-\sqrt{4}} +.....+ \dfrac{1}{\sqrt{8}+\sqrt{9}} = 2$
\end{enumerate}
Q.4 If $x = 2+\sqrt{3}$ find the value of $x^2+\dfrac{1}{x^2}$\\
Q.5 If $x = 3+2\sqrt{2}$ find the value of $x^2+\dfrac{1}{x^2}$\\
Q.6 If $x = \dfrac{\sqrt{3}+\sqrt{2}}{\sqrt{3}-\sqrt{2}}$ and $y = x = \dfrac{\sqrt{3}-\sqrt{2}}{\sqrt{3}+\sqrt{2}}$, find $x^2+y^2$\\
Q.7 If $x = \dfrac{1}{2-\sqrt{3}}$ find the value of $x^3-2x^2-7x+5$\\
Q.8 Arrange in ascending order.
\begin{enumerate}[(i)]
\item $(10)^\frac{1}{2}, (14)^\frac{1}{3}, (200)^\frac{1}{6}$
\item $(17)^\frac{1}{5}, (100)^\frac{1}{10}, (200)^\frac{1}{20}$
\end{enumerate}
%%%%%%%%%%%%%%%%%%%%%%%%% END SECTION 6 %%%%%%%%%%%%%%%%%%%%%%%%
%{\Large \textbf{Answer}}
%\setcounter{section}{1}
%\section{section}
%Q.1
%\begin{multicols}{5}
%\begin{enumerate}[a.]
%\item $0.\overline{6}$
% \item $-0.\overline{4}$
% \item $-0.\overline{13}$  
 %\item $-1.\overline{6692307}$ 
 %\item $0.\overline{437}$ 
%\end{enumerate}
%\end{multicols}
%\section{Section}
%%%%%%%%%%%%%%%%%%%%%%%%% SECTION 7 %%%%%%%%%%%%%%%%%%%%%%%%%%
\section{Algebraic Identities}
Here are some identities which is to be used frequently\\\\
%\begin{center}
$(a+b)^2 = a^2+2ab+b^2$\\
$(a-b)^2 = a^2-2ab+b^2$\\
$a^2-b^2 = (a+b)(a-b)$\\
$(a+b)^3 = a^3+b^3+3ab(a+b) = a^3+b^3+3a^2b+3ab^2$\\
$(a-b)^3 = a^3-b^3-3ab(a-b) = a^3-b^3-3a^2b+3ab^2$\\
$a^3+b^3 = (a+b)(a^2-ab+b^2)$\\
$a^3-b^3 = (a-b)(a^2+ab+b^2)$\\
$(a+b+c)^2 = a^2+b^2+c^2+2ab+2bc+2ca$\\
%\end{center}
\subsection{Exercises}
Q.1 Find the product using identities
\begin{multicols}{2}
\begin{enumerate}[(i)]
\item $(7x+y)(7x-y)$
\item $(x-1)(x+1)(x^2+1)(x^4+1)$
\item $\Big(x-\dfrac{1}{x}\Big)\Big(x+\dfrac{1}{x}\Big)\Big(x^2+\dfrac{1}{x^2}\Big)\Big(x^4+\dfrac{1}{x^4}\Big)$
\item $103\times93$
\item $(97)^2$
\item $0.54\times0.54 - 0.46\times0.46$
\item $103\times107$
\item $104\times96$
\end{enumerate}
\end{multicols}
\noindent Q.2 If $x-\dfrac{1}{x} = 6$
\begin{multicols}{2}
\begin{enumerate}[(i)]
\item $x^2+\dfrac{1}{x^2}$
\item $x^4+\dfrac{1}{x^4}$
\end{enumerate}
\end{multicols}
\noindent Q.3 If $\Big(x^2+\dfrac{1}{x^2}\Big) = 27$, find the value of
\begin{multicols}{2}
\begin{enumerate}[(i)]
\item $x+\dfrac{1}{x}$
\item $x-\dfrac{1}{x}$
\end{enumerate}
\end{multicols}
\noindent Q.4 If $x+y = 12$ and $xy = 32$, find the value of $x^2+y^2$.\\
Q.5 If $4x^2+y^2 = 40$ and $xy = 6$, find the value of $2x+y$.\\
Q.6 Simplify
\begin{multicols}{2}
\begin{enumerate}[(i)]
\item $(2x+3y)^3$
\item $(4x+2y)^3-(4x-2y)^3$
\end{enumerate}
\end{multicols}
\noindent Q.7 If $x+\dfrac{1}{x} = 7$, find the value of $x^3+\dfrac{1}{x^3}$\\
Q.8 If $x+\dfrac{1}{x} > 0$ and $x^2+\dfrac{1}{x^2} = 7$, find the value of $x^3+\dfrac{1}{x^3}$.\\
Q.9 If $x^2+\dfrac{1}{x^2}= 03$, find the value of $x^3-\dfrac{1}{x^3} = 7$, find the value of $x^3-\dfrac{1}{x^3}$ (Take $x-\dfrac{1}{x} < 0)$ \\
Q.10 Evaluate using identities.
%\begin{multicols}
\begin{enumerate}[(i)]
\item $23^3-17^3$
\item $29^3-11^3$
\end{enumerate}
%\end{multicols}
\noindent Q.11 If $a-b = 4$ and $ab = 45$, find the value of $a^3-b^3$\\
Q.12 If $a+b = 10$ and $ab = 21$, find the value of $a^3+b^3$\\
Q.13 If $a+b = 7$ and $ab = 12$, find the value of $(a^2-ab+b^2)$\\
Q.14 If $a^2+b^2+c^2 = 20$ and $a+b+c = 0$, find the value of $ab+bc+ca$\\
Q.15 If $a+b+c = 9$ and $ab+bc+ca = 40$, find $a^2+b^2+c^2$\\
Q.16 If $a^2+b^2+c^2 = 250$ and $ab+bc+ca = 3$, find a+b+c\\
Q.17 Expand
\begin{multicols}{3}
\begin{enumerate}[(i)]
\item $(x+2y+4z)^2$
\item $(2a-3b-c)^2$
\item $(-2x+3y+2z)^2$
\end{enumerate}
\end{multicols}
\noindent Q.18 Simplify
%\begin{multicols}
\begin{enumerate}[(i)]
\item $(a+b+c)^2+(a-b-c)^2$
\item $(a+b+c)^2-(a-b-c)^2 $
\end{enumerate}
%\end{multicols}
\noindent Q.19 Prove that\\\\
$a^3+b^3+c^3-3abc = (a+b+b)(a^2+b^2+c^2-ab-bc-ca)$\\\\
(This is a standard identity, students are advised to remember this).\\\\
Hint : $a^3+b^3 = (a+b)^3-3ab(a+b)$\\\\
Q.20 By using the last exercise
%\begin{multicols}{1}
\begin{enumerate}[(i)]
\item Show that if $a+b+c = 0$ then $a^3+b^3+c^3 = 3abc$ but converse is false.
\item If $a+b+c = 6$ then $ab+bc+ca = 11$, find the value of $a^3+b^3+c^3-3abc$.
\end{enumerate}
%\end{multicols}
\noindent Q.21 By using the first part of last exercise find the value of
\begin{multicols}{2}
\begin{enumerate}[(i)]
\item $(28)^3+(-15)^3+(-13)^3$
\item $(-12)^2+7^3+5^3$
\item $30^3+20^3-50^3$
\item $\dfrac{(a-b)^3+(b-c)^3+(c-a)^3}{(a-b)(b-c)(c-a)}$
\item $\dfrac{(a^2-b^2)^3+(b^2-c^2)^3+(c^2-a^2)^3}{(a-b)(b-c)(c-a)}$
\end{enumerate}
\end{multicols}
%%%%%%%%%%%%%%%%%%%%%%%%%%%55%%%%%% END SECTION 7 %%%%%%%%%%%%%%%%%%%%%%%%%%%
\section{Primes and Composites}
\begin{tcolorbox}
Definition : A prime number (or simply prime) is a natural number $p >1$ whose only positive divisors 1 and $p$.
\end{tcolorbox}
\begin{tcolorbox}
Definition : A natural number greater than 1 which is not prime is called
a composite number (or simply composite).
\end{tcolorbox}
\noindent \fbox{Fact : 1 is the only natural number which is neither prime nor
composite.}\\\\
\fbox{Fact : If none of the prime number upto
$\sqrt{n}$ divides $n$ then $n$ is a prime number. (Try to prove
it yourself).}\\\\
\subsection{Exercises}
Q.1 {Identify which of the following is prime?}
\begin{multicols}{3}
\begin{enumerate}[(i)]
\item 197
\item 297
\item 499
\item 553
\item 773
\end{enumerate}
\end{multicols}
\noindent Q.2 {Find the smallest composite number that has no prime
factors less than 10.}\\
Q.3 {What is the largest two-digit prime number whose digits
are also each prime?}\\
Q.4 {How many prime number are perfect cubes?}\\
Q.5 {The number 13 is prime. If you reverse the digits you
also obtain a prime number, 31.
What is the larger of the pair of primes that satisfies this condition
and has a sum of 110?}\\
Q.6 {What is the smallest prime divisor of 101729 + 729101?}\\
Q.7 {A group of 25 coins is arrange into %three piles such that
each pile contains a different
prime number of coins. What is the greatest number of coins possible in
any of the three
piles?}\\
Q.8 {Is 9409 prime?}\\
Q.9 {What are the 5 smallest prime numbers greater than 1000?}\\
Q.10 {What is the first year in the twenty-first century that
is a prime number?}\\
%%%%%%%%%%%%%%%%%%% Exercises %%%%%%%%%%%%%%%%%%%%%%%%%%%%%%%%
\newpage
\begin{center}
\concept{Exercises}
\end{center}
\level{Level 1}
\begin{multicols}{2}
\begin{enumerate}
\item Which of following fractions lie between $\dfrac{1}{4}$ and $\dfrac{1}{5}$
\begin{multicols}{4}
\begin{enumerate}[A]
\item $\dfrac{7}{33}$
\item $\dfrac{4}{11}$
\item $\dfrac{13}{57}$
\item $\dfrac{7}{17}$
\end{enumerate}
\end{multicols}
\begin{multicols}{2}
\begin{enumerate}[(a)]
\item A and B
\item A and C
\item B,C and D
\item A,B and D
\end{enumerate}
\end{multicols}
\item Express $0.\overline{34}+0.3\overline{4}$ asa single decimal.
\begin{multicols}{2}
\begin{enumerate}[(a)]
\item $0.67\overline{88}$
\item $0.68\overline{78}$
\item $0.6\overline{89}$
\item $0.6\overline{87}$
\end{enumerate}
\end{multicols}
\item If $\sqrt{5^n} = 125$, then $5^{\sqrt[n]{64}} = \rule{1cm}{0.15mm}$
%%%%%%%%%%%%%%%%%%%%%%%%%5
\begin{multicols}{2}
\begin{enumerate}[(a)]
\item 25
\item $\dfrac{1}{125}$
\item 625
\item $\dfrac{1}{25}$
\end{enumerate}
\end{multicols}
\item If $x^4+1 = 1297$ and $y^4-1 = 2400$, then $y^2-x^2 = \rule{1cm}{0.15mm}$
\begin{multicols}{2}
\begin{enumerate}[(a)]
\item 10
\item 25
\item 13
\item 43
\end{enumerate}
\end{multicols}
\item What is the value of $4^{(2x-2)}$, if $(16)^{2x+3} = (64)^{x+3}$?
\begin{multicols}{2}
\begin{enumerate}[(a)]
\item 64
\item 256
\item 32
\item 512
\end{enumerate}
\end{multicols}
\item Which of the following have equal values?\\
(where $x \in \mathbb{R}) \rule{1cm}{0.15mm}$
\begin{multicols}{2}
\begin{enumerate}[(a)]
\item $9^{\frac{x}{2}}, 24^{\frac{x}{3}}$
\item $(256)^{\frac{4}{x}}, (4^3)^{\frac{4}{x}}$
\item $(343)^{\frac{x}{3}}, (7^4)^{\frac{x}{12}}$
\item $(36^2)^{\frac{2}{7}}, (6^3)^{\frac{2}{7}}$
\end{enumerate}
\end{multicols}
\item The expression $\Bigg(\sqrt{5}-\sqrt{3}\Bigg)\Bigg(\sqrt{7}-\sqrt{2}\Bigg)$ when simplified becomes a
\begin{multicols}{2}
\begin{enumerate}[(a)]
\item simple surd.
\item mixed surd.
\item compound surd.
\item binomial surd
\end{enumerate}
\end{multicols}
\item If $m$ and $n$ are positive integers, then for a positive number a,$ \Bigg\{\sqrt[m]{\Big(\sqrt[n]{a}\Big)}\Bigg\}^{mn} = \rule{1cm}{0.15mm}$
\begin{multicols}{2}
\begin{enumerate}[(a)]
\item $a^{mn}$
\item $a$
\item $a^{m/n}$
\item 1
\end{enumerate}
\end{multicols}
\item If $2^{-m}\times\dfrac{1}{2^m} = \dfrac{1}{4}$, then $\dfrac{1}{14}\Bigg\{(4^m)^{\frac{1}{2}}+\Big(\dfrac{1}{5^m}\Big)^{-1}\Bigg\} = \rule{1cm}{0.15mm}$
\begin{multicols}{2}
\begin{enumerate}[(a)]
\item $\dfrac{1}{2}$
\item 2
\item 4
\item $\dfrac{-1}{4} $
\end{enumerate}
\end{multicols}
\item The surds $\sqrt{2}, \sqrt[3]{3}$ and $\sqrt[5]{5}$, in their descending order are
\begin{multicols}{2}
\begin{enumerate}[(a)]
\item $\sqrt[3]{3},\sqrt[5]{5},\sqrt{2}$
\item $\sqrt{2},\sqrt[3]{3},\sqrt[5]{5}$
\item $\sqrt{2},\sqrt[5]{5},\sqrt[3]{3}$
\item $\sqrt[3]{3},\sqrt{2},\sqrt[5]{5}$
\end{enumerate}
\end{multicols}
\item $2[(16-15)^{-1}+25(13-8)^{-2}]^{-1}+(1024)^0 = \rule{1cm}{0.15mm}$
\begin{multicols}{2}
\begin{enumerate}[(a)]
\item 2
\item 3
\item 1
\item 5
\end{enumerate}
\end{multicols}
\item If $x = 2$ and $y = 4$, then $\Big(\dfrac{x}{y}\Big)^{x-y}+\Big(\dfrac{y}{x}\Big)^{y-x} = \rule{1cm}{0.15mm}$
\begin{multicols}{2}
\begin{enumerate}[(a)]
\item 4
\item 8
\item 12
\item 2
\end{enumerate}
\end{multicols}
\item In which of the following pairs of surds are the given two surds similar?
\begin{multicols}{2}
\begin{enumerate}[(a)]
\item $\sqrt{5},7\sqrt{5}$
\item $\sqrt[3]{7},\sqrt[2]{7}$
\item $\sqrt{7},\sqrt{28}$
\item Both $(a)$ and $(b)$
\end{enumerate}
\end{multicols}
\item Which of the following is the greatest?
\begin{multicols}{2}
\begin{enumerate}[(a)]
\item $7^2$
\item $(49)^{\frac{3}{2}}$
\item $\Big(\dfrac{1}{343}\Big)^{\frac{-1}{3}}$
\item $(2401)^{\frac{-1}{4}}$
\end{enumerate}
\end{multicols}
\item $(\sqrt[6]{5})(\sqrt[3]{2})(\sqrt{3})(\sqrt[12]{6}) = \rule{1cm}{0.15mm}$
\begin{multicols}{2}
\begin{enumerate}[(a)]
\item $\sqrt[12]{1749600}$
\item $\sqrt[3]{2}\times\sqrt[12]{109350}$
\item $\sqrt[12]{177960}$
\item Both $(a)$ and $(b)$
\end{enumerate}
\end{multicols}
\item If $p = 3$ and $q = 2$,then $(3p-4q)^{q-p}\div(4p-3q)^{2q-p} = \rule{1cm}{0.15mm}$ 
\begin{multicols}{2}
\begin{enumerate}[(a)]
\item 1
\item 6
\item $\dfrac{1}{6}$
\item $\dfrac{2}{3}$
\end{enumerate}
\end{multicols}
\item $\Bigg[\dfrac{(32)^{0.2}+(81)^{0.25}}{(256)^{0.5}-(121)^{0.5}}\Bigg] = \rule{1cm}{0.15mm}$
\begin{multicols}{2}
\begin{enumerate}[(a)]
\item 2
\item 5
\item 1
\item 11
\end{enumerate}
\end{multicols}
\item The smallest among the surds $\sqrt{10}-\sqrt{5},\sqrt{19}-\sqrt{14},\sqrt{22}-\sqrt{17}$ and $\sqrt{8}-\sqrt{3}$ is
\begin{multicols}{2}
\begin{enumerate}[(a)]
\item $\sqrt{10}-\sqrt{5}$
\item $\sqrt{19}-\sqrt{14}$
\item $\sqrt{22}-\sqrt{17}$
\item $\sqrt{8}-\sqrt{3}$
\end{enumerate}
\end{multicols}
\item If $\sqrt{m} = \sqrt{a}+\sqrt{c}$ and $\sqrt{m}, \sqrt{a}$ and $\sqrt{c}$ are three surds
%\begin{multicols}{2}
\begin{enumerate}[(a)]
\item $\sqrt{m}$ is dissimilar to $\sqrt{a}$ and $\sqrt{c}$
\item $\sqrt{a}$ and $\sqrt{c}$  are similar to $\sqrt{m}$
\item only $\sqrt{a}$ is similar to $\sqrt{m}$
\item None of these
\end{enumerate}
%\end{multicols}
\item The surd obtained after rationalizing the numerator of $\dfrac{4-\sqrt{25-a}}{a-9}$ is equal to
\begin{enumerate}[(a)]
\item $\dfrac{a-9}{4-\sqrt{25-a}}$
\item $\dfrac{1}{4-\sqrt{25-a}}$
\item $\dfrac{1}{(a-9)(4-\sqrt{25-a})}$
\item $\dfrac{1}{4+\sqrt{25-a}}$
\end{enumerate}
\item If $\sqrt{13-x\sqrt{10}} = \sqrt{8}+\sqrt{5}$, then what is the value of $x $?
\begin{multicols}{2}
\begin{enumerate}[(a)]
\item -5
\item -6
\item -4
\item -2
\end{enumerate}
\end{multicols}
\item If the surds $\sqrt[4]{4}, \sqrt[6]{5}, \sqrt[8]{6}$ and $12\sqrt{8}$ are aranged in ascending order from left to right, then the third surd from the left is
\begin{multicols}{2}
\begin{enumerate}[(a)]
\item $\sqrt[12]{8}$
\item $\sqrt[4]{4}$
\item $\sqrt[8]{6}$
\item $\sqrt[6]{5}$
\end{enumerate}
\end{multicols}
\item $\sqrt{11\sqrt{11\sqrt{
11...4 \text{ terms}}}} = \rule{1cm}{0.15mm}$
\begin{multicols}{2}
\begin{enumerate}[(a)]
\item $\sqrt[16]{11^5}$
\item  $\sqrt[16]{11}$
\item  $\sqrt[16]{11^{14}}$
\item  $\sqrt[16]{11^{15}}$
\end{enumerate}
\end{multicols}
\item If $\dfrac{5+\sqrt{3}}{2+\sqrt{3}} = x+\sqrt{3}$, then $(x,y)$ is
\begin{multicols}{2}
\begin{enumerate}[(a)]
\item (13,-7)
\item  (-13,7)
\item  (-13,-7)
\item  (13,7)
\end{enumerate}
\end{multicols}
\item The simplest form of $\sqrt{125}+\sqrt{125}-\sqrt{845}$ is 
\begin{multicols}{2}
\begin{enumerate}[(a)]
\item $\sqrt{15}$
\item  $2\sqrt{5}$
\item   $-\sqrt{5}$
\item $-2\sqrt{5}$ 
\end{enumerate}
\end{multicols}
\item Which of following statement is true?
\begin{enumerate}[I.]
\item If $x$ is a conjugate surd of $y$, then $x$ can be a $RF$ of $y$
\item If $x$ is a $RF$ of $y$, then $x$ need not to be the conjugate of $y$
\end{enumerate}
\begin{multicols}{2}
\begin{enumerate}[(a)]
\item Only I
\item Only II
\item Both I and II
\item Neither I nor II
\end{enumerate}
\end{multicols}
\item If $\dfrac{3-2\sqrt{5}}{6-\sqrt{5}} = a+b\sqrt{5}$ where $a$ and $b$ are rational numbers, then what are the values of $a$ and $b$?
\begin{multicols}{2}
\begin{enumerate}[(a)]
\item $\dfrac{8}{35},\dfrac{-9}{35}$
\item $\dfrac{8}{31},\dfrac{-9}{31}$
\item $\dfrac{-8}{31},\dfrac{9}{31}$
\item $\dfrac{-8}{35},\dfrac{9}{35}$
\end{enumerate}
\end{multicols}
\item If $\dfrac{3^{5x}\times(81)^2\times6561}{3^{2x}} = 3^7$, then $x = \rule{1cm}{0.15mm}$
\begin{multicols}{2}
\begin{enumerate}[(a)]
\item 3
\item -3
\item $\dfrac{1}{3}$
\item $\dfrac{-1}{3}$
\end{enumerate}
\end{multicols}
\item If $\sqrt{2^n} = 1024$, then $3^{2\Big(\frac{n}{4}-4\Big)} = \rule{1cm}{0.15mm}$
\begin{multicols}{2}
\begin{enumerate}[(a)]
\item 3
\item -3
\item 27
\item 81
\end{enumerate}
\end{multicols}
\item If $\Bigg[\bigg\{\Big(\dfrac{1}{7^2}\Big)^{-2}\bigg\}^{\frac{-1}{3}}\Bigg]^{\frac{2}{4}} = 7^m$, then $m = \rule{1cm}{0.15mm}$
\begin{multicols}{2}
\begin{enumerate}[(a)]
\item $\dfrac{-1}{3}$
\item $\dfrac{1}{4}$
\item -3
\item 2
\end{enumerate}
\end{multicols}
%%%%%%%%%%%%%%%%%%%%%%%%%%%%%%%%%%%%%%%%%%%%%%%%%%%%%%%%%
\level{Level 2}
\item $\Bigg[\bigg\{\Big(\dfrac{1}{x^{a^2-b^2}}\Big)^{\frac{1}{a-b}}\bigg\}^{a+b}\Bigg]^{\frac{1}{(a+b)^2}} = \rule{1cm}{0.15mm}$
\begin{multicols}{2}
\begin{enumerate}[(a)]
\item $x^2$
\item $\dfrac{1}{x}$
\item $7^3$
\item $\dfrac{1}{x^2}$
\end{enumerate}
\end{multicols}
\item If $\dfrac{2^{m+n}}{2^{m-n}}$ and $a = 2^{\frac{1}{10}}$, then $\dfrac{(a^{2m+n-p})^2}{(a^{m-2n+2p})^{-1}} = \rule{1cm}{0.15mm}$
\begin{multicols}{2}
\begin{enumerate}[(a)]
\item 2
\item $\dfrac{1}{4}$
\item 9
\item $\dfrac{1}{8}$
\end{enumerate}
\end{multicols}
\item $\Bigg[(p^{-1}+q^{-1})(p^{-1}-q^{-1})+\bigg(\dfrac{1}{p^{-1}}-\dfrac{1}{q^{-1}}\bigg)\bigg(\dfrac{1}{p^{-1}}+\dfrac{1}{q^{-1}}\bigg)     \Bigg](pq)^2$
\begin{multicols}{2}
\begin{enumerate}[(a)]
\item $(pq)^2$
\item -1
\item $-(pq)^2$ 
\item 1
\end{enumerate}
\end{multicols}
\item If $x = \dfrac{2}{\sqrt{10}-\sqrt{8}}, y = \dfrac{2}{\sqrt{10}-2\sqrt{2}}$, then $(x-y)^2 = \rule{1cm}{0.15mm}$
\begin{multicols}{2}
\begin{enumerate}[(a)]
\item $4\sqrt{2}$
\item 32
\item  $8\sqrt{2}$
\item 64
\end{enumerate}
\end{multicols}
\item If $a = \sqrt{6}-\sqrt{3}, b = \sqrt{3}-\sqrt{2}$ and $c = \sqrt{2}-\sqrt{6}$, then find the value of $a^3+b^3+c^3-2abc$.
\begin{enumerate}[(a)]
\item $3\sqrt{2}-5\sqrt{3}-\sqrt{6}$
\item $3\sqrt{2}-5\sqrt{3}-\sqrt{6}$
\item  $3\sqrt{2}-4\sqrt{3}+\sqrt{6}$
\item $3\sqrt{2}+4\sqrt{3}+\sqrt{6}$
\end{enumerate}
\item $\sqrt{\dfrac{81}{64}\sqrt{\dfrac{81}{64}\sqrt{\dfrac{81}{64}\sqrt{\dfrac{81}{64}}....\infty}}} = $
\begin{multicols}{2}
\begin{enumerate}[(a)]
\item $\dfrac{81}{64}$
\item $\dfrac{9}{8}$
\item $\dfrac{3}{2}$
\item $\dfrac{3}{2\sqrt{2}}$
\end{enumerate}
\end{multicols}
\item If $a^p = b^q = c^r = abc$, then $pqr = \rule{1cm}{0.15mm} $
\begin{enumerate}[(a)]
\item $p^2q+q^2r$
\item $pq+qr+pr$
\item $(pq+qr+rp)^2$
\item $pq(qr+rp)$
\end{enumerate}
\item The value of $\Bigg[(23+2^2)^{\frac{2}{3}}+(140-29)^{\frac{1}{2}}\Bigg]^2$ is $\rule{1cm}{0.15mm}$
\begin{multicols}{2}
\begin{enumerate}[(a)]
\item 196
\item 289
\item 324
\item 400
\end{enumerate}
\end{multicols}
\item If $x = \sqrt{6}+\sqrt{5}$, then $x^2+\dfrac{1}{x^2}-2 = \rule{1cm}{0.15mm}$
\begin{multicols}{2}
\begin{enumerate}[(a)]
\item $2\sqrt{6}$
\item $2\sqrt{5}$
\item 24
\item 20
\end{enumerate}
\end{multicols}
\item $\sqrt{6+\sqrt{6+\sqrt{6+......\infty}}}$ is equal to $\rule{1cm}{0.15mm}$
\begin{multicols}{2}
\begin{enumerate}[(a)]
\item -3
\item 3
\item 6
\item 2
\end{enumerate}
\end{multicols}
\item Simplify\\
$\dfrac{1}{\sqrt{19-\sqrt{360}}}-\dfrac{1}{\sqrt{21-\sqrt{440}}}+\dfrac{2}{\sqrt{20-\sqrt{396}}} =$ %\rule{1cm}{0.15mm}$
\begin{multicols}{2}
\begin{enumerate}[(a)]
\item 1
\item 2
\item 0
\item Nome of these
\end{enumerate}
\end{multicols}
\item If $a = \sqrt{17}-\sqrt{16}$ and $b = \sqrt{16}\sqrt{15}$, then 
\begin{multicols}{2}
\begin{enumerate}[(a)]
\item $a < b$
\item $a > b$
\item $a = b$
\item None of these
\end{enumerate}
\end{multicols}
\item $\Bigg(\sqrt[6]{15-2\sqrt{56}}\Bigg).\Bigg(\sqrt[3]{\sqrt{7}+2\sqrt{56}}\Bigg) = \rule{1cm}{0.15mm}$
\begin{multicols}{2}
\begin{enumerate}[(a)]
\item 0
\item 1
\item -1
\item 2
\end{enumerate}
\end{multicols}
\item $\sqrt{\sqrt{63}+\sqrt{56}} = \rule{1cm}{0.15mm}$
\begin{multicols}{2}
\begin{enumerate}[(a)]
\item $\sqrt[4]{7}(\sqrt{3}+\sqrt{5})$
\item $\sqrt[4]{7}(\sqrt{3}+1)$
\item $\sqrt[4]{7}(\sqrt{3}+\sqrt{5})$
\item $\sqrt[4]{7}(\sqrt{2}+1)$
\end{enumerate}
\end{multicols}
%%%%%%%%%%%%5
\item If $\dfrac{\sqrt{7}+2\sqrt{3}}{2\sqrt{7}-\sqrt{5}} = \dfrac{c+\sqrt{p+\sqrt{q}+\sqrt{r}}}{23} (p < q < r)$, where $p, q, r$ are rational numbers, then $q+r-p = \rule{1cm}{0.15mm}$
%\begin{multicols}{2}
\begin{enumerate}[(a)]
\item 361
\item 302
\item 418
\item 426
\end{enumerate}
%\end{multicols}
\item The following are the steps involved in finding the value of $x-y$ from $\dfrac{8-\sqrt{5}}{8+\sqrt{5}} = x-y\sqrt{40}$. Arrange them in sequential order.
%%%%%%%%%%%%%%%%%%%%%%5
\begin{enumerate}[(A)]
\item $\dfrac{13-2\sqrt{40}}{8-5} = x-y\sqrt{40}$
\item $\dfrac{(\sqrt{8})^2+(\sqrt{5})^2-2(\sqrt{8})(\sqrt{5})}{(\sqrt{8})^2+(\sqrt{5})^2} = x-y\sqrt{40}$
\item $x-y = \dfrac{11}{3}$
\item $x = \dfrac{13}{3}$ and $\dfrac{2}{3}$
\item $\dfrac{(\sqrt{8}-\sqrt{5})(\sqrt{8}+\sqrt{5})}{(\sqrt{8}+\sqrt{5})(\sqrt{8}-\sqrt{5})} = x-y\sqrt{40}$
\end{enumerate}
\begin{enumerate}[(a)]
\item EABDC
\item EBADC
\item ABDEC
\item DEBAC
\end{enumerate}
\item The following are the steps involved in finding the least among $\sqrt{3}, \sqrt[3]{4}$ and $\sqrt[6]{15}$. Arrange them in sequential order.
\begin{enumerate}[(A)]
\item $\sqrt[6]{5}$ is the smallest.
\item $3^\frac{1}{2} = 3^\frac{3}{6}, 4^\frac{2}{6}, 15^\frac{1}{6} = 15^\frac{1}{6}$
\item The LCM of the denominators of the exponents is 6.
\item $\sqrt{3} = 3^\frac{1}{2}, 3\sqrt{4} = 4^\frac{1}{3}, \sqrt{3} = 15^\frac{1}{6}$
\item $\sqrt{3} = \sqrt[6]{27}, \sqrt[3]{4} = \sqrt[6]{16}, \sqrt[6]{15} = \sqrt[6]{15}$
\end{enumerate}
\begin{enumerate}[(a)]
\item DCABE
\item DABEB
\item DCBEA
\item DBCAE
\end{enumerate}
\item $y = 3-\sqrt{8}$, then $\Big(y-\dfrac{1}{y}\Big)^2  = $
\begin{multicols}{2}
\begin{enumerate}[(a)]
\item 9
\item 81
\item 4
\item 32
\end{enumerate}
\end{multicols}
\item The following steps are involved in finding the value of $a+b$ from $\dfrac{2+\sqrt{3}}{2-\sqrt{3}} = a+b\sqrt{3}$. Arrange them in sequential order.
\begin{enumerate}[(A)]
\item $\dfrac{2^2+(\sqrt{3})^2+2\times2\times\sqrt{3}}{2^2-(\sqrt{3})^2} = a+b\sqrt{3}$
\item $a+b = 7+4 = 11$
\item $\dfrac{\Big(2+\sqrt{3}\Big)\Big(2+\sqrt{3}\Big)}{\Big(2-\sqrt{3}\Big)\Big(2+\sqrt{3}\Big)} = a+b\sqrt{3}$
\item $\dfrac{7+4\sqrt{3}}{4-3} = a+b\sqrt{3}$
\item $a =7$ and $b = 4$
\end{enumerate}
\begin{enumerate}[(a)]
\item CDAEB
\item CAEBD
\item CADEB
\item CEDAB
\end{enumerate}
\item The following are the steps involved in finding the greatest among $\sqrt[3]{2}, \sqrt[6]{3}$ and $\sqrt{6}$. Arrange them in sequential order.
\begin{enumerate}[(A)]
\item The LCM of the denominators of the exponents is 6.
\item $\sqrt[6]{216}$ i.e $\sqrt{6}$ is the greatest.
\item $3\sqrt{2} = 2^{\frac{1}{3}}, 6\sqrt{3} = 3^{\frac{1}{6}}, \sqrt{6} = 6^{\frac{1}{2}}$
\item $2^{\frac{1}{3} = 2^{\frac{2}{6}}}, 3^{\frac{1}{6} = 3^{\frac{1}{6}}}, 6^{\frac{1}{2} = 6^{\frac{3}{6}}}$
\item $\sqrt[3]{2} = \sqrt[6]{4}, \sqrt[6]{3} = \sqrt[6]{3}, \sqrt{6} = \sqrt[6]{216}$
\end{enumerate}
\begin{enumerate}[(a)]
\item CADEB
\item CDABE
\item DCAEB
\item DACBE
\end{enumerate}
\item If $x = \dfrac{1}{\sqrt{3}+2}$, then $\Big(x+\dfrac{1}{x}\Big)^2 = \rule{1cm}{0.15mm}$
\begin{multicols}{2}
\begin{enumerate}[(a)]
\item 16
\item 3
\item 12
\item 6
\end{enumerate}
\end{multicols}
\item If $\sqrt[x]{3} \times \sqrt[y]{5} = 10125$, then $12xy = \rule{1cm}{0.15mm}$
\begin{multicols}{2}
\begin{enumerate}[(a)]
\item 1
\item $\dfrac{1}{3}$
\item 2
\item $\dfrac{1}{2}$
\end{enumerate}
\end{multicols}
\item If $x = \dfrac{1}{5+2\sqrt{6}}$, then $x^2-10x+1 = \rule{1cm}{0.15mm}$
\begin{multicols}{2}
\begin{enumerate}[(a)]
\item 1
\item -1
\item 0
\item 10
\end{enumerate}
\end{multicols}
\item If $x = \dfrac{2}{\sqrt{3}-\sqrt{5}}$ and $y = \dfrac{2}{\sqrt{3}+\sqrt{5}}$, then $x+y = \rule{1cm}{0.15mm}$
\begin{multicols}{2}
\begin{enumerate}[(a)]
\item 3
\item $4\sqrt{3}$
\item $-2\sqrt{3}$
\item 6
\end{enumerate}
\end{multicols}
\item $\dfrac{3}{7}$ lies between the fractions $\rule{1cm}{0.15mm}$
\begin{multicols}{2}
\begin{enumerate}[(a)]
\item $\dfrac{4}{9}, \dfrac{5}{9}$
\item $\dfrac{43}{99}, \dfrac{4}{9}$
\item $\dfrac{42}{99}, \dfrac{4}{9}$
\item $\dfrac{41}{99}, \dfrac{42}{99}$
\end{enumerate}
\end{multicols}
%%%%%%%%%%%%%%%%%%%%%%%%%%%%%%%%%%%%%%%%%%%%%%%%%%5
\level{Level 3}
\item If $\sum_{k=4}^{143} \dfrac{1}{\sqrt{k}+\sqrt{k+1}} = a-\sqrt{b}$, then $a$ and $b$ respectively are
\begin{multicols}{2}
\begin{enumerate}[(a)]
\item 10 and 0
\item -10 and 4
\item 1o and 4
\item -10 and 0
\end{enumerate}
\end{multicols}
\item The surd $\dfrac{12}{3+\sqrt{5}+2\sqrt{2}}$, after rationalizing the denominators becomes
\begin{multicols}{2}
\begin{enumerate}[(a)]
\item $\sqrt{5}-\sqrt{2}+\sqrt{10}+1$
\item $\sqrt{5}+\sqrt{10}+\sqrt{2}+1$
\item $\sqrt{10}+\sqrt{2}+\sqrt{5}+1$
\item $\sqrt{5}-\sqrt{10}-\sqrt{2}-1$
\end{enumerate}
\end{multicols}
\item If $A^\frac{1}{A} = B^\frac{1}{B} = C^\frac{1}{C}, A^{BC}+B^{AC}+C^{AB} = 729 $.\\
Which of the following equals $A^\frac{1}{A} ?$ 
\begin{multicols}{2}
\begin{enumerate}[(a)]
\item $\sqrt[ABC]{81}$
\item $\sqrt{2}$
\item $\sqrt[ABC]{27}$
\item $\sqrt[ABC]{9}$
\end{enumerate}
\end{multicols}
\item If $x = \dfrac{1}{2-\sqrt{3}}$, the value of $x^3-2x^2-7x+10$ is equal to
\begin{multicols}{2}
\begin{enumerate}[(a)]
\item $2+\sqrt{3}$
\item 10
\item  $7+2\sqrt{3}$
\item 8
\end{enumerate}
\end{multicols}
\item If $x = 1+5^{\frac{1}{3}}+5^{\frac{2}{3}}$, then find the value of $x^3-3x^2-12x+6.$
\begin{multicols}{2}
\begin{enumerate}[(a)]
\item 22
\item 20
\item 16
\item 14
\end{enumerate}
\end{multicols}
\item $\dfrac{4}{\sqrt{10-2\sqrt{21}}} = \rule{1cm}{0.15mm}$
\begin{multicols}{2}
\begin{enumerate}[(a)]
\item $\dfrac{1}{4}\Big(\sqrt{7}+\sqrt{3}\Big) $
\item $\dfrac{1}{4}\Big(\sqrt{7}-\sqrt{3}\Big)$
\item $\sqrt{7}+\sqrt{3}$
\item $\sqrt{7}-\sqrt{3}$
\end{enumerate}
\end{multicols}
\item If $y = 3^\frac{1}{3}+3$, then $y^3-9y^2+27y = \rule{1cm}{0.15mm}$
\begin{multicols}{2}
\begin{enumerate}[(a)]
\item 27
\item -27
\item -30
\item 30
\end{enumerate}
\end{multicols}
\item $\dfrac{1}{\sqrt{8+2+\sqrt{15}}} = \rule{1cm}{0.15mm}$
\begin{multicols}{2}
\begin{enumerate}[(a)]
\item $\dfrac{1}{2}\Big(\sqrt{5}+\sqrt{3}\Big)$
\item  $\dfrac{1}{2}\Big(\sqrt{5}-\sqrt{3}\Big)$
\item  $\dfrac{1}{2}\Big(\sqrt{5}+1\Big)$
\item  $\dfrac{1}{2}\Big(\sqrt{5}-1\Big)$
\end{enumerate}
\end{multicols}
\item If $x = 2^\frac{1}{3}-2$, then $x^3+6x^2+12x = \rule{1cm}{0.15mm}$
\begin{multicols}{2}
\begin{enumerate}[(a)]
\item 6
\item -6
\item 8
\item -8
\end{enumerate}
\end{multicols}
\item $\dfrac{3}{\sqrt{19-2\sqrt{88}}}-\dfrac{8}{\sqrt{14+2\sqrt{33}}} = \rule{1cm}{0.15mm}$
\begin{enumerate}[(a)]
\item ${\sqrt{19+2\sqrt{33}}}$
\item ${\sqrt{14-2\sqrt{88}}}$
\item ${\sqrt{11+2\sqrt{24}}}$
\item ${\sqrt{11-2\sqrt{55}}}$
\end{enumerate}
\item $\sqrt{\sqrt[x]{2^{x}\sqrt[x^2]{3^{x^3}\sqrt[x^3]{6^{x^6}\sqrt[x^4]{9^{x^{10}}}}}}} = \rule{1cm}{0.15mm}$
\begin{multicols}{2}
\begin{enumerate}[(a)]
\item 18
\item 54
\item 24
\item 36
\end{enumerate}
\end{multicols}
\item $\sqrt{7+2\sqrt{6}}+\sqrt{7-2\sqrt{6}} = \rule{1cm}{0.15mm}$
\begin{multicols}{2}
\begin{enumerate}[(a)]
\item 14
\item $\sqrt{6}$
\item $2\sqrt{6}$
\item 7
\end{enumerate}
\end{multicols}
\item $\sqrt{3^2\sqrt{9^2\sqrt{(81)^2\sqrt{(16)^{16}}}}} = \rule{1cm}{0.15mm}$
\begin{multicols}{2}
\begin{enumerate}[(a)]
\item $6\times2^4$
\item $3^3\times2$
\item $6^3\times2^3$
\item $6^3\times2$
\end{enumerate}
\end{multicols}
\item $\sqrt[6]{15-2\sqrt{56}}. \sqrt[3]{\sqrt{7}+2\sqrt{2}} = \rule{1cm}{0.15mm}$
\begin{multicols}{2}
\begin{enumerate}[(a)]
\item 0
\item $\sqrt{2}$
\item 1
\item $6\sqrt{2}$
\end{enumerate}
\end{multicols}
\item If $p = 7-4\sqrt{3}$, then $\dfrac{p^2+1}{7p} =  \rule{1cm}{0.15mm}$
\begin{multicols}{2}
\begin{enumerate}[(a)]
\item 2
\item 1
\item 7
\item $\sqrt{3}$
\end{enumerate}
\end{multicols}
%%%%%%%%%%%%%%%%%%%%%%%%%%%%%%%%%%%
\end{enumerate}
\end{multicols}
%%%%%%%%%%%%%%%%%% END LEVELS QUESTIONS %%%%%%%%%%%%%%%%%%%%%
\vspace{26mm}
\begin{center}
\concept{Answers}
\end{center}
\level{4.1 Exercises}
\begin{enumerate}[Q.1]
\item 
\begin{multicols}{5}
\begin{enumerate}[(i)]
\item 521
\item 91
\item 723
\item 895
\item 253
\item 68.3
\item 21.31
\item 2.64575...
\item 2.82842...
\item 3.31662...
\end{enumerate}
\end{multicols}
\end{enumerate}
%==========================================================
\level{6.3 Exercises}
\begin{enumerate}[Q.1]
\item 
\begin{multicols}{3}
\begin{enumerate}[(i)]
\item $\dfrac{2}{7}\sqrt{7}$
\item $\dfrac{2}{9}\sqrt{3}$
\item $\dfrac{5}{4}(3+\sqrt{5})$
\item $\dfrac{31+10\sqrt{6}}{19}$
\item $\dfrac{47+21\sqrt{5}}{4}$
\item $\dfrac{18-2\sqrt{10}-4\sqrt{6}-3\sqrt{15}}{19}$
\end{enumerate}
\end{multicols}
\item 
\begin{multicols}{3}
\begin{enumerate}[(i)]
\item $a = 2, b = -1$
\item $a = 8, b = 3$
\item $a = 11, b = -6$
\item $a = 1, b = 27$
\item $a = 4, b = 1$
\item $a = 2, b = -\dfrac{5}{6}$
\end{enumerate}
\end{multicols}
\begin{multicols}{5}
\item 
\item 14
\item 34
\item $5+2\sqrt{6}$
\item 3
\end{multicols}
\begin{multicols}{3}
\item 
\begin{enumerate}[(i)]
\item $(14)^{\frac{1}{3}}, (200)^{\frac{1}{6}}, (10)^{\frac{1}{2}}$
\item $(200)^{\frac{1}{20}}, (100)^{\frac{1}{10}}, (17)^{\frac{1}{5}}$
\end{enumerate}
\end{multicols}
\end{enumerate}
%===================  END OUTER ENUMERATE=============
\level{7.1 Exercises}
\begin{enumerate}[Q.1]
\item 
\begin{multicols}{3}
\begin{enumerate}[(i)]
\item $49x^2-y^2$
\item $x^8-1$
\item $x^8-\dfrac{1}{x^8}$
\item 9991
\item 9409
\item 0.08
\item 11021
\item 9984
\end{enumerate}
\end{multicols}
\item 
\begin{multicols}{4}
\begin{enumerate}[(i)]
\item 34
\item 1154
\end{enumerate}%%%
\item 
\begin{enumerate}[(i)]
\item $\pm \sqrt{29}$
\item $\pm 5$
\end{enumerate}
\item 80
\item $\pm 8$
\end{multicols}
\begin{multicols}{2}
\item 
\begin{enumerate}[(i)]
\item $8x^3+27y^3+36x^2y+54xy^2$
\item $16y^3+192x^2y$
\end{enumerate}
\end{multicols}
\begin{multicols}{4}
\item 322
\item 18
\item -756
\item 
\begin{enumerate}[(i)]
\item 7254
\item 23058
\end{enumerate}
\end{multicols}
\begin{multicols}{4}
\item 604
\item 370
\item 13
\item  -10
\end{multicols}
\begin{multicols}{4}
\item 1
\item $\pm 16$
\item 
\end{multicols}
\begin{multicols}{2}
\begin{enumerate}[(i)]
\item $x^2+4y^2+16z^2+4xy+16yz+8zx$
\item $4a^2+9b^2+c^2-12ab+6bc-4ac$
\end{enumerate}
\end{multicols}
\item 
\begin{multicols}{2}
\begin{enumerate}[(i)]
\item $2(a^2+b^2+c^2+2abc)$
\item 4a(b+c)
\end{enumerate}
\end{multicols}
\begin{multicols}{4}
\item  (ii) 18
\item 
\begin{enumerate}[(i)]
\item 1638
\item -1260
\item -90000
\end{enumerate}
\end{multicols}
\end{enumerate}

%%%%%%%%%%%%%%%%%%%%%%%%%%%% 8.1 exercise%%%%%%%%%%%%%%%
\level{8.1 Exercises}
\begin{enumerate}[Q.1]
\begin{multicols}{4}
\item 197, 499 and 773 are primes.
\item 121
\item 73
\end{multicols}
 \begin{multicols}{2}
\item No prime number is perfect cube.
\item 73
\end{multicols}
\begin{multicols}{4}
\item 2
\item 17
\item No
\end{multicols}
\begin{multicols}{4}
\item 1009, 1013, 1019, 1021, 1031
\item 2003
\end{multicols}
\end{enumerate}

%==============================================
\level{Level 1}
\begin{enumerate}[1.]
\begin{multicols}{10}
\item (b)
\item (d)
\item (a)
\item (c)
\item (b)
\item (a)
\item (c)
\item (b)
\item (a)
\item (d)
\item (a)
\item (b)
\item (d)
\item (b)
\item (d)
\item (c)
\item (c)
\item (c)
\item (b)
\item (d)
\item (c)
\item (d)
\item (d)
\item (a)
\item (c)
\item (c)
\item (b)
\item (b)
\item (b)
\item (a)
\end{multicols}
%========================================
\level{Level 2}
\begin{multicols}{10}
\item (b)
\item (a)
\item (b)
\item (b)
\item (c)
\item (a)
\item (b)
\item (d)
\item (d)
\item (b)
\item (c)
\item (a)
\item (b)
\item (d)
\item (a)
\item (b)
\item (c)
\item (d)
\item (c)
\item (a)
\item (a)
\item (a)
\item (c)
\item (c)
\item (c)
\end{multicols}
\level{Level 3}
\begin{multicols}{10}
\item (a)
\item (b)
\item (b)
\item (d)
\item (a)
\item (c)
\item (d)
\item (b)
\item (b)
\item (c)
\item (a)
\item (c)
\item (d)
\item (c)
\item (a)
\end{multicols}
\end{enumerate}
%%%%%%%%%%%%%%%%%%%%%%%%%%%%%%%%%%%%%%%%%%%%
