\documentclass[a4paper,twocolumn]{article}
\usepackage[svgnames]{xcolor}
\usepackage{fancyhdr}
\usepackage{titling}
	\newcommand{\HorRule}{\color{DarkGoldenrod} \rule{\linewidth}{1pt}}

\pretitle{\vspace{-70pt} \begin{flushleft}  \HorRule \fontsize{50}{50}
 \color{DarkBlue}  \usefont{OT1}{phv}{b}{sl}  } % Horizontal rule
%before the title

\title{Second Life} % Your article title

\posttitle{\end{flushleft}} % Whitespace under the title

\preauthor{\begin{flushleft}\large  \usefont{OT1}{phv}{b}{sl} 
\color{DarkBlue}} % Author font configuration

\author{Manpreet Dhiman, } % Your name

\postauthor{\footnotesize  \usefont{OT1}{phv}{m}{sl} \color{Black} %
%Configuration for the institution name
G.N.D.E.C. % Your institution

\end{flushleft}\HorRule} % Horizontal rule after the title

%%%%%%%%%%%%%%%%%%%%% Initial Letter %%%%%%%%%%%%%%%%%%%%%%%%%%
\usepackage{lettrine} % Package to decorate the first letter of the
%paragraph
\usepackage{fix-cm}
\newcommand{\initial}[1]{ % Defines the command and style for the
% first letter
\lettrine[lines=3,lhang=0.6,nindent=0em]{
\color{DarkGoldenrod}
{\textsf{#1}}}{}}

\begin{document}
\maketitle
\initial{S}\textbf{econd Life is a virtual world in three dimensions 
that you can enter through the Internet. Users can travel around, talk
or communicate with others, buy land and houses, build cars, parks or
any other object you can think of. You can go shopping, attend colleges
and universities or go to concerts and other events. You can do almost
anything you can do in the real world. Second Life was created in 2003
by a San Francisco company called Linden Lab.} 

\section{The world of Second Life}

When you sign up for a free account you become a resident. It gives you
the right to create an avatar and travel around in Second Life. In
order to view this virtual world you must download a special viewer and
install it on your computer.

The Second Life world is made up of different regions. The mainland is
the biggest region. It is owned by Linden Lab itself. There are also
other areas, owned by private people or other companies.

A membership fee of ten dollars a month lets you to buy your own land.
By becoming a paying member you also get some virtual money that you can
spend. You also get help if you have problems or run into trouble.

 

New residents who enter Second Life for the first time start on “
Orientation Island ”. Users learn the rules of the online world, how to
navigate from one place to another and how to communicate with others.
They are also given a tour of interesting places to visit.

If you have your own land in Second Life you can use it in any way you
want. You can build houses or other objects on it, let it to others or
even design your own landscape. 

\section{Avatars}
Every resident in Second Life has their own avatar. They can
choose ready made ones or design avatars of their own.

You can choose hair and skin colour, body parts or clothing from an
inventory. An avatar does not have to be human. It can also be an
animal, plant or even a robot, like in Star Wars.

An avatar can be changed at any time and many people really have fun
constantly changing their appearance. 

\section{Travel and communication in Second Life}

Avatars can move around in Second Life in three ways. The easiest way is
walking. Clicking a button lets you fly over landscapes like Superman
and gives you a bird’s eye view of the online world.

The fastest method of travelling over long distances is teleporting. You
open up a map and click on the place you want to go to. However, you
can’t go everywhere you want. Some places are off limits, and sometimes
you must ask owners for permission to travel across their land.

Communication in Second Life is easy. Residents can talk to each other
with microphones attached to their computers Or they may write messages
in specially designed chat boxes. 

\section{Creating Objects}

Everything that you see in Second Life was created by one of its
members. New residents can practice making objects in the so-called
sandbox. You can design your own buildings, houses, cars or other
objects by using building blocks that come in all shapes. Complicated
or even animated objects can be made by using a special programming
language in Second Life.

\section{Population}

According to Linden Lab there are about 10 million user accounts in
Second Life. This number, however, does not show the real number of
people who take part because a user may have more than one account. Most
of them do not log on to the virtual world regularly and only 10% of all
registered users come back after their first visit.

Residents of Second Life come from all areas of society. Doctors,
students and big company bosses take part, as well as factory workers or
housewives. About 60% of all residents are men.

When you live in Second Life you must keep some rules in mind:
\begin{itemize}
\item  It is forbidden to discriminate against other races , cultures or
religions.
\item   Violence, using bad language or running around naked is not
allowed in areas that are marked as safe.
\item  You are not allowed to reveal personal information about someone
else.
\end{itemize}
Young people are not allowed into Second Life. Youngsters between 13
and 17 must log on to their own virtual world called Teen Second Life.
It is a separate 3D world which works in much the same way as Second
Life for adults. Additional helpers show youngsters around and help them
navigate. 

\section{Economy}

Second Life has its own money called the Linden dollar. Residents can go
to a bank and convert real money into Linden dollars. One US dollar buys
you about 267 dollars in the virtual world. Members who pay a monthly
fee get about 300 Linden dollars a week to spend.

Not everything in Second Life can be bought with Linden dollars.
Sometimes you will have to pay with real money, for example, if you buy
a larger area of land.

\section{Second Life and the real world}

More and more people and even companies around the world are finding out
that they can improve their popularity by taking part in Second Life.
Some companies organize staff meetings, others organize public events
for virtual users.

Real life colleges and universities have started to use Second Life to
hold classes. Politicians are using Second Life for their campaigns and
to become popular. Even TV stations have their own presence in Second
Life.   

Although some Internet experts say that Second Life gives people freedom
and lets them do things that they can’t do in real life they do point
out some dangers.

\begin{itemize}
\item  Second Life is not that easy to learn. A new member may spend 
a lot of time in front of their computer.
\item   The virtual world might make you forget your real friends.
\item   Some experts say that players sometimes get confused between 
Second Life and the real world.
\end{itemize}

\end{document}
