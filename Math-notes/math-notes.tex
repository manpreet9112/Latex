\documentclass[a4paper,10pt]{article}
\usepackage[left=1.0cm, right=1.0cm, top=1.5cm, bottom=2.0cm]{geometry}
\usepackage{fancyhdr}
\usepackage{amsmath}
\usepackage{enumerate}
\usepackage{tikz}
\usepackage{amsfonts}
\usepackage{graphicx}
\usepackage{multicol}
\usepackage[edges]{forest}
\usetikzlibrary{arrows.meta}
%\pagestyle{fancy}
%\usepackage{fancyheadings}
%\fancyhead[C]{Aaegis}
%\fancyfoot[C]{\textbf {Aaegis Academy, Opp. Atul Sweets, Jehan Circle,
%Gangapur
%Road, Nasik}}
%\lhead{Board Level Exercise}
%\rhead{AAEGIS ACADEMY}
%\usepackage[usenames, dvipsnames]{color}
%===================================
\usepackage{draftwatermark}
\SetWatermarkText{\includegraphics{images/logo.png}}
\SetWatermarkLightness{0.1}
\SetWatermarkScale{1.0}
%============= MACRO ====================================%
\newcommand{\questiontype}[1]{\begin{center}\hrule \vspace{2mm}
\Huge{\textbf{#1}}\vspace{2mm}\hrule \end{center}}
%\renewcommand\thesubsubsection{\normalsize {Q. \arabic{subsubsection}}}
%\setcounter{subsubsection}{0}
\newcommand{\pic}[1]{

\begin{center}

\includegraphics[width=0.8\linewidth]{#1.png}

\end{center}

}
\newcommand{\picx}[1]{

\begin{center}

\includegraphics[width=0.3\linewidth]{#1.png}

\end{center}

}

\graphicspath{{images/}}
\begin{document}
\questiontype{Number System}
\section{Classification of Numbers}
Undoubtedly numbers are the cornerstone of mathematics. There is a huge
literature on the
development of numbers but that is beyond our scope. We directly jump to
the classification
of numbers. Remember this classification looks so obvious or trivial
today but humans took centuries to reach here. Reader should appreciate
this.\\
%%%%%%%%%%%%%% CLASSIFICATION TREE %%%%%%%%%%%%%%%%%%%%%%%%
\begin{center}
\begin{forest}
  forked edges,
  for tree={edge+={-Latex}},
  [Numbers
    [Real Numbers
      [Rational Numbers
        [Fractions]
        [Integers
        [Negative Integers]
        [Non-negative Integer (Whole Numbers)
        [\big\{0\big\}]
        [Positive Integers
(Natural numbers)]
        ]
        ]
      ]
      [Irrational Numbers]
    ]
    [Imaginary Number]
  ]
\end{forest}
\end{center}
%%%%%%%%%%%%%%%%%%%%%%%%%%%%%%%%%%%%%%%%%%%%%%%%%%%%%%%%%
 Here are some notations to denote the set of numbers :\\\\
 $\mathbb{N}$ - Set of Natural Numbers = $\big\{1,2,3,....\big\}$\\
 $\mathbb{N}_0$ or $\mathbb{W}$ - Set of whole numbers = $\big\{0,1,2,...\big\}$\\
$\mathbb{Z}$- Set of integers = $\big\{....-5,-4,...-1,0,1,2...\big\}$\\
$\mathbb{Q}$ - Set of rational numbers =
$\big\{...-5,\dfrac{-4}{7},\dfrac{-1}{2},0,\dfrac{1}{2},\dfrac{1}{3},\dfrac{5}{7},4,...\big\}$\\\\
%* All integers are rational numbers\\\\
$\mathbb{Q`}$ or $\mathbb{Q}^c$ - Set of irrational Numbers =
$\big\{...-\sqrt{11},-\sqrt{5},-\sqrt{7},-\dfrac{1}{\sqrt{3}}....\sqrt{5},...\big\}$\\
$\mathbb{R}$ - Set of real Number.\\\\
A number is said to be rational if it can be written as $\dfrac{p}{q}$ form where
p, q are integers. For example
$\dfrac{1}{2},\dfrac{-1}{5},\dfrac{1}{-5},-\dfrac{2}{7},-1,0$,
$4,-\dfrac{7}{8}$
are rational numbers. A real number is said to be irrational if it is
not rational. For example
$\sqrt{2},\sqrt{3},\sqrt{5},-\sqrt{2},\dfrac{1}{\sqrt{2}}$,
$-\dfrac{\sqrt{3}}{\sqrt{2}}$
are irrational number. (It is very interesting to prove that
$\sqrt{2},\sqrt{3}$ etc cannot be written as $\dfrac{p}{q}$ form. We
shall prove this shortly). 
\subsection{Exercises}
Q.1 {Find five rational numbers between}
\begin{multicols}{5}
\begin{enumerate}[a.]
\item 1 and 10
\item 11 and 12
\item $\dfrac{1}{3}$ and $\dfrac{1}{2}$
\item $-\dfrac{1}{5}$ and $-\dfrac{1}{7}$
\item $-\dfrac{3}{5}$ and $\dfrac{3}{5}$  
\end{enumerate}
\end{multicols}
\noindent Q.2 {State true or false}
\begin{enumerate}[a.]
\item Every rational number is an integer.
\item Every natural number is an integer.
\item Every integer is an rational number.
\item 5 is an non-negative number.
\item 0 is an non-negative integer.
\item Every irrational number is an rational number.
\item $-\dfrac{1}{7}$ and 7 are rational numbers.
\end{enumerate}
Q.3 {Express the following in decimal by log division}
\begin{multicols}{5}
\begin{enumerate}[a.]
\item $\dfrac{19}{5}$
\item $\dfrac{3}{15}$
\item $-\dfrac{19}{2}$
\item $\dfrac{2157}{625}$
\item $-\dfrac{17}{8}$
\item $\dfrac{327}{500}$
\end{enumerate}
\end{multicols}
\noindent Q.4 {Show that the decimal expansion of following rational
numbers are non-terminating and
repeating.}
\begin{multicols}{4}
\begin{enumerate}[a.]
 \item $\dfrac{8}{3}$
 \item $\dfrac{2}{11}$
 \item $\dfrac{1}{7}$
 \item $-\dfrac{16}{45}$
\end{enumerate}
\end{multicols}
%=========== End section 1==============%
\section{Non-terminating and recurring rational number}
There are rational numbers such that when we try to express them in
decimal form by
division method, we find that no matter how long we divide there is
always a remainder. In
other words, the division process never comes to an end. This is due to
the reason that in the
division process the remainder starts repeating after a certain number
of steps. In such cases,
a digit of a block of digits repeats itself. For example,
0.3333...,0.1666666...,
0.123123123...., 1.2692307692307692307...etc. Such decimals are called
non-terminating
repeating decimals or non-terminating recurring decimals. These decimal
numbers are
represented by putting a bar over the first block of the repeating part
and omit the other
repeating blocks. Thus, we write 0.33333... = 0.3, 0.16666... =
$\overline{0.16}$, 0.123123123 ..... = $\overline{0.123}$ and
1.26923076923076292307... = $\overline{1.2692307}$.\\\\
Fact : Every non terminating and recurring number is a rational
number.\\\\
1) Express the following rational numbers as decimals
\begin{multicols}{5}
\begin{enumerate}[(i)]
 \item $\dfrac{2}{3}$
 \item $-\dfrac{4}{5}$
 \item $-\dfrac{2}{15}$
 \item $-\dfrac{22}{13}$
 \item $\dfrac{437}{999}$
 \item $\dfrac{33}{26}$
\end{enumerate}
\end{multicols}
\noindent 2) Look at several examples of rational numbers in the form
$\dfrac{p}{q} (\neq 0)$, where p and q are integers with no common
factors other than 1 and having terminating decimal representations. Can
you guess what property q must satisfy?
%===================== SECTION 2 END======================%
\section{Conversion and non-terminating recurring rational numbers in
the form $\dfrac{p}{q}$}
1) Express $0.\overline{4}$ in the form $\dfrac{p}{q}$\\
2) Express each of the following decimals in the form 
\begin{multicols}{2}
\begin{enumerate}[(i)]
\item $0.\overline{35}$
\item $0.\overline{585}$
\end{enumerate}
\end{multicols}
\noindent 3) Express the following decimals in the form $\dfrac{p}{q}$
\begin{multicols}{3}
\begin{enumerate}[(i)]
\item $0.3\overline{2}$
\item $0.12\overline{3}$
\item $0.003\overline{52}$
\end{enumerate}
\end{multicols}
\noindent 4) Express each of the following decimals in the form $\dfrac{p}{q}$
\begin{multicols}{7}
\begin{enumerate}[(i)]
\item $0.\overline{4}$
\item $0.\overline{37}$
\item $0.\overline{54}$
\item $0.\overline{621}$
\item $125.\overline{3}$
\item $4.\overline{7}$
\item $0.\overline{47}$
\end{enumerate}
\end{multicols}
\newpage
\begin{center}
\textbf{\Large Tricky Tip}
\hrule
\vspace{3mm}
$\dfrac{p}{q} = \frac{\text{Complete Number - Number formed by non -
repeating digits}}{\text{No.of 9 as no.of repeating digits after that write
no.of 0 as no.of non - repeating digits}}$
\begin{enumerate}[(i)]
\item  $0.\overline{57} = \dfrac{57-0}{99} = \dfrac{57}{99}$
\item $0.\overline{347} = \dfrac{347-3}{990} = \dfrac{349}{990}$
\item $0.53\overline{763} = \dfrac{53763-53}{99900} =
\dfrac{53710}{99900} = \dfrac{5371}{9990}$
\end{enumerate}
\hrule
\end{center}
%%%%%%%%%%%%%%%%%%% END SECTION 3 %%%%%%%%%%%%%%%%%%%%%%%%%%
\section{Irrational Numbers (finding the square root by
division method)}
A number which is not rational is known as a irrational number. In other
words, a number which can’t be written as $\dfrac{p}{q}$ form.\\
Fact : Every non-terminating and non-recurring number is irrational
number.\\\\
$\sqrt{2},\sqrt{3},\sqrt{7},\pi$ are rational number.
Let’s find the value of $\sqrt{2},\sqrt{3}$ by division method.\\\\
1.\\
%%%%%%%%%%%%%%%%%%%%%%%%% DIVISION METHOD SUM 1 %%%%%%%%%%%%%%%
\begin{center}
\begin{tabular}{l|l } 
 
       & 1.4142135 \\\cline{2-2}
1      & 2.00000000000000 \\
       & 1 \\\cline{2-2} 
24     & 100\\ 
       &{\hspace{2mm}96}\\\cline{2-2} 
28     &{\hspace{4mm}400}\\
       &{\hspace{4mm}281}\\\cline{2-2}
2824   &{\hspace{4mm}11900}\\
       &{\hspace{4mm}11296}\\\cline{2-2} 
28282  &{\hspace{8mm}60400}\\ 
       &{\hspace{8mm}56564}\\\cline{2-2} 
282841 &{\hspace{10mm}383600}\\ 
       &{\hspace{10mm}282841}\\\cline{2-2}
2828423&{\hspace{10mm}10075900}\\ 
       &{\hspace{10mm}8485269}\\\cline{2-2}
28284265&{\hspace{10mm}159063100}\\
       &{\hspace{10mm}141421325}\\\cline{2-2} 
28284270&{\hspace{12mm}17611775}                              
\end{tabular}
\end{center}
%%%%%%%%%%%%%%%%%%%%%%%%%%%%%%%%%%%%%%%%%%%%%%%%%%%%%%%%%%
$\Rightarrow \sqrt{2} = 1.4142135$\\\\
2.\\
%%%%%%%%%%%%%%%%%%%% DIVISION METHOD SUM 2 %%%%%%%%%%%%%%%
\begin{center}
\begin{tabular}{l|l } 
 
       & 1.732050807\\\cline{2-2}
1      &3.00 00 00 00 00 00 00 00 \\
       & 1 \\\cline{2-2} 
27     & 200\\ 
       &{\hspace{2mm}189}\\\cline{2-2} 
343     &{\hspace{4mm}1100}\\
       &{\hspace{4mm}1029}\\\cline{2-2}
3462   &{\hspace{8mm}7100}\\
       &{\hspace{8mm}6924}\\\cline{2-2} 
346405  &{\hspace{8mm}1760000}\\ 
       &{\hspace{8mm}1732025}\\\cline{2-2} 
34641008 &{\hspace{12mm}279750000}\\ 
       &{\hspace{12mm}277128064}\\\cline{2-2}
3464101607&{\hspace{16mm}26219360000}\\ 
       &{\hspace{16mm}24248711249}\\\cline{2-2}
       &{\hspace{18mm}1970648751}\\
                                   
\end{tabular}
\end{center}
%%%%%%%%%%%%%%%%%%%%%%%%%%%%%%%%%%%%%%%%%%%%%%%%%%%%%%
$\Rightarrow \sqrt{3} = 1.732050807$
%%%%%%%%%%%%%%%%%%%%%%%%%%%% End section 4 %%%%%%%%%%%%%%%%%%%
\section{Proof by Contradiction (Irrationality of $\sqrt{2}$)}
We will discuss this topic more thoroughly in the chapter ‘Mathematical
Logic’. Currently
our main motive is to prove the irrationality of $\sqrt{2} i.e \sqrt{2}$
can’t be written as $\dfrac{p}{q}$ form. This
method ‘proof by contradiction’ is used frequently in calculus, one of
the most interesting
branch of mathematics. In this method we assume the negation of
something which is to beproved then use deduce a contradiction which
implies that our assumption was wrong. Now let’s prove that $\sqrt{2}$ is an
irrational number.\\\\
Assume that $\sqrt{2}$ is an rational number it means it can be written
as $\dfrac{p}{q}$
form. Let\\
$\sqrt{2} = \dfrac{p}{q}$\\
Where p, q are co-prime ($p, q$ have not common factor) Now we have\\
$\alpha = \dfrac{p^2}{q^2}$\\
$p^2 = \alpha q^2$\\\\
It shows that $p^\alpha$ is an even integer, since $p$ is an integer it must be
an even integer. So take $p = \alpha m$ for some integer m. Now we
have\\\\
$4 m^\alpha = 2 \alpha^2$\\
$2 m^\alpha = q^2$\\\\
It shows $q^\alpha$ is an even integer hence $q$ is an even integer which
contradicts our assumption
that $p, q$ are co-prime. Hence our assumption was wrong that
$\sqrt{2}$ is an irrational number.
$\sqrt{2}$ is an rational number.\\\\ 
\setcounter{subsection}{4}
\subsection{Exercises}
%\large{\textbf{5.1 Exercises}}\\
%\usecounter{subsubsection}
Q.1 Prove that $\sqrt{3}$ is an irrational number\\
Q.2 Prove that $\sqrt{5}$ is an irrational number\\
Q.3 Prove that $\sqrt{n}$ is an irrational number if n is not a
perfect square.\\
Q.4 Prove that $\sqrt{3}+\sqrt{5}$ is an irrational number.\\
Q.5 Using the fact that negative of rational number is also a
rational number prove that
negative of irrational number is an irrational number.\\
Q.6 Prove that the sum of rational number and irrational
number is an irrational number.\\
Q.7 Prove a disprove that sum of two irrational number is an
irrational number.\\
Q.8 Prove that $\sqrt{ab}$ lies between two positive numbers between a
and b $(a < b)$. Hence find 2
irrational numbers between\\
\begin{multicols}{3}
\begin{enumerate}[(i)]
\item 2 and 3
\item $\sqrt{2}$ and $\sqrt{3}$
\item 2 and $\sqrt{5}$
\end{enumerate}
\end{multicols}
%%%%%%%%%%%%%%%%%%%%%%%%%%%%% END SECTION 5 %%%%%%%%%%%%%%%%%%%%
\section{Primes and Composites}
Definition : A prime number (or simply prime) is a natural number $p >1$
whose only positive divisors 1 and $p$.\\\\
Definition : A natural number greater than 1 which is not prime is called
a composite number (or simply composite).\\\\
Fact : 1 is the only natural number which is neither prime nor
composite.\\
Fact : If none of the prime number upto
$\sqrt{n}$ divides $n$ then $n$ is a prime number. (Try to prove
it yourself).\\\\
%\Large{\textbf{6.1 Exercises}\\
%\usecounter{subsubsection}
\subsection{Exercises}
Q.1 {Identify which of the following is prime?}
\begin{multicols}{3}
\begin{enumerate}[(i)]
\item 197
\item 297
\item 499
\item 553
\item 773
\end{enumerate}
\end{multicols}
\noindent Q.2 {Find the smallest composite number that has no prime
factors less than 10.}\\
Q.3 {What is the largest two-digit prime number whose digits
are also each prime?}\\
Q.4 {How many prime number are perfect cubes?}\\
Q.5 {The number 13 is prime. If you reverse the digits you
also obtain a prime number, 31.
What is the larger of the pair of primes that satisfies this condition
and has a sum of 110?}\\
Q.6 {What is the smallest prime divisor of 101729 + 729101?}\\
Q.7 {A group of 25 coins is arrange into three piles such that
each pile contains a different
prime number of coins. What is the greatest number of coins possible in
any of the three
piles?}\\
Q.8 {Is 9409 prime?}\\
Q.9 {What are the 5 smallest prime numbers greater than 1000?}\\
Q.10 {What is the first year in the twenty-first century that
is a prime number?}\\\\
%%%%%%%%%%%%%%%%%%%%%%%%%%% END SECTION 6 %%%%%%%%%%%%%%%%%%%%%%%%
\Large{\textbf{Answer}}\\\\
\large{\textbf{Section 2}\\
1)
\begin{multicols}{5}
\begin{enumerate}[a.]
\item $0.\overline{6}$
 \item $-0.\overline{4}$
 \item $-0.\overline{13}$  
 \item $-1.\overline{6692307}$ 
 \item $0.\overline{437}$ 
\end{enumerate}
\end{multicols}
\noindent  \large{\textbf{Section 3}\\
%\usecounter{subsubsection}
%\subsubsection{}
1)
Solution :\\
Let $x = 0.\overline{4} = 0.444$\\
10 $x = 4.444....$\\
10 $ x - x = 4$\\
$9x  = 4 = x = \dfrac{4}{9}$\\\\
%\subsubsection{}
2)  Solution :\\\\
(i) Let $x = 0.\overline{35}$\\
$x = 0.353535......(i)$\\\\
Here, we have two repeating digits after the decimal point. So, we
multiply sides of (i) by $10^2 = 100$ to get\\
$100 x = 35.3535$... (ii)\\
Subtracting (i) from (ii), we get\\
$100x –x = (35.3535...) – (0.3535...)$\\
$99x = 35$\\
$X = \dfrac{35}{99}$\\
Hence, $0.\overline{35} = \dfrac{35}{99}$\\\\
%%%%%%%%%%%%%%%%%%%
(ii) Let $x = 0.\overline{585}$\\
$x = 0.585585585...$\\\\
Here, we have three repeating digits after the decimal point. So, we
multiply both sides of (i) by $10^3 =  1000$ to get\\\\
$1000 x = 585.585585...$\\
Subtracting (i) from (ii), we get\\
$1000 x –x = (585.585585...)- (0.585585585...)$\\
$1000 x – x = 585$\\
$999x = 585$\\
$ x = \dfrac{585}{999}$\\\\
3) Solution\\\\
(i) Let $x = 0.3\overline{2}$ \\\\
Clearly, there is just one digit on the right side of the decimal point
which is without bar. So,
we multiply both sides of $x$ by 10 so that only the repeating decimal is
left on the right side of
the decimal point.\\\\
 i.e $10 x = 3.\overline{2}$\\
$\Rightarrow 10x = 3+0.\overline{2}$\\
$\Rightarrow 10x = 3+\dfrac{2}{9}$\\
$\Rightarrow 10x = \dfrac{9\times3+2}{9} \Rightarrow 10x = \dfrac{29}{9}
\Rightarrow x = \dfrac{29}{90}$\\\\
(ii) Let $x = 0.12\overline{3}$\\\\
Clearly, there are two digits on the righ side of the decimal point
which are without bar. So,
we multiply both sides of $x$ by 102 = 100 so that only the repeating
decimal is left on the right
side of the decimal point.\\\\
i.e $100x = 12.\overline{3}$\\
$\Rightarrow 100x = 12+0.\overline{3}$\\  
$\Rightarrow 100x = 12+\dfrac{3}{9}$\\
$\Rightarrow 100x = \dfrac{12\times9+3}{9}$\\
$\Rightarrow 100x = \dfrac{108+3}{9} \Rightarrow 100x = \dfrac{111}{9}
\Rightarrow x = \dfrac{111}{900}$\\\\
(iii) Let $x = 0.003\overline{52}$\\\\
Clearly, there are three digits on the right side of the decimal point
which are without bar. So,
we multiply both sides of $x$ by 103= 1000 so that only the repeating
decimal is left on the
right side of the decimal point.\\\\
i.e $1000x = 3.\overline{52}$\\
$\Rightarrow  1000x =  3+0.52$\\
$\Rightarrow  1000x = 3+\dfrac{52}{99}$\\
$\Rightarrow  1000x = \dfrac{3\times99+52}{99}$\\
$\Rightarrow  1000x = \dfrac{297+52}{99} \Rightarrow  1000x =
\dfrac{349}{99} \Rightarrow  x = \dfrac{340}{99000}$\\\\
4) Solution
\begin{multicols}{6}
\begin{enumerate}[(i)]
\item $\dfrac{4}{9}$
\item  $\dfrac{37}{99}$
\item $\dfrac{6}{11}$
\item $\dfrac{23}{37}$
\item $\dfrac{376}{3}$
\item $\dfrac{43}{9}$
\item $\dfrac{43}{90}$
\end{enumerate}
\end{multicols}
%=============================e
\end{document}
