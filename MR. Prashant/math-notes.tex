\documentclass[a4paper,10pt]{article}
\usepackage[left=1.0cm, right=1.0cm, top=1.5cm, bottom=2.0cm]{geometry}
\usepackage{fancyhdr}
\usepackage{amsmath}
\usepackage{enumerate}
\usepackage{tikz}
\usepackage{amsfonts}
\usepackage{graphicx}
\usepackage{multicol}
\usepackage[edges]{forest}
\usetikzlibrary{arrows.meta}
%\pagestyle{fancy}
%\usepackage{fancyheadings}
%\fancyhead[C]{Aaegis}
%\fancyfoot[C]{\textbf {Aaegis Academy, Opp. Atul Sweets, Jehan Circle,
%Gangapur
%Road, Nasik}}
%\lhead{Board Level Exercise}
%\rhead{AAEGIS ACADEMY}
%\usepackage[usenames, dvipsnames]{color}
%===================================
\usepackage{draftwatermark}
\SetWatermarkText{\includegraphics{images/logo.png}}
\SetWatermarkLightness{0.1}
\SetWatermarkScale{1.0}
%============= MACRO ====================================%
\newcommand{\questiontype}[1]{\begin{center}\hrule \vspace{2mm}
\Huge{\textbf{#1}}\vspace{2mm}\hrule \end{center}}
%\renewcommand\thesubsubsection{\normalsize {Q. \arabic{subsubsection}}}
%\setcounter{subsubsection}{0}
\newcommand{\pic}[1]{

\begin{center}

\includegraphics[width=0.8\linewidth]{#1.png}

\end{center}

}
\newcommand{\picx}[1]{

\begin{center}

\includegraphics[width=0.3\linewidth]{#1.png}

\end{center}

}


\begin{document}
\questiontype{Number System}
\section{Section : Classification of Numbers}
Undoubtedly numbers are the cornerstone of mathematics. There is a huge
literature on the
development of numbers but that is beyond our scope. We directly jump to
the classification
of numbers. Remember this classification looks so obvious or trivial
today but humans took centuries to reach here. Reader should appreciate
this.\\\\
Here are some notations to denote the set of numbers :\\\\
 N - Set of Natural Numbers = ${1,2,3,....}$\\
 N$_0$ or W - Set of whole numbers = ${0,1,2,...}$\\
Z - Set of integers = ${....-5,-4,...-1,0,1,2...}$\\
Q - Set of rational numbers =
${...-5,\dfrac{-4}{7},\dfrac{-1}{2},0,\dfrac{1}{2},\dfrac{1}{3},\dfrac{5}{7},4,...}$\\\\
* All integers are rational numbers\\\\
Q' or Q$^c$ - Set of irrational Numbers =
$...-\sqrt{11},-\sqrt{5},-\sqrt{7},-\dfrac{1}{\sqrt{3}....\sqrt{5},...}$\\
R - Set of real Number.\\\\
A number is said to be rational if it can be written as $\dfrac{p}{q}$ form where
p, q are integers. For example
$\dfrac{1}{2},\dfrac{-1}{5},\dfrac{1}{-5},-\dfrac{2}{7},-1,0$,
$4,-\dfrac{7}{8}$
are rational numbers. A real number is said to be irrational if it is
not rational. For example
$\sqrt{2},\sqrt{3},\sqrt{5},-\sqrt{2},\dfrac{1}{\sqrt{2}}$,
$-\dfrac{\sqrt{3}}{\sqrt{2}}$
are irrational number. (It is very interesting to prove that
$\sqrt{2},\sqrt{3}$ etc cannot be written as $\dfrac{p}{q}$ form. We
shall prove this shortly). 
\subsection{Exercises}
\subsubsection{Find five rational numbers between}
\begin{multicols}{5}
\begin{enumerate}[a.]
\item 1 and 10
\item 11 and 12
\item $\dfrac{1}{3}$ and $\dfrac{1}{2}$
\item $-\dfrac{1}{5}$ and $-\dfrac{1}{7}$
\item $-\dfrac{3}{5}$ and $\dfrac{3}{5}$  
\end{enumerate}
\end{multicols}
\subsubsection{State true or false}
\begin{enumerate}[a.]
\item Every rational number is an integer.
\item Every natural number is an integer.
\item Every integer is an rational number.
\item 5 is an non-negative number.
\item 0 is an non-negative integer.
\item Every irrational number is an rational number.
\item $-\dfrac{1}{7}$ and 7 are rational numbers.
\end{enumerate}
\subsubsection{Express the following in decimal by log division}
\begin{multicols}{5}
\begin{enumerate}[a.]
\item $\dfrac{19}{5}$
\item $\dfrac{3}{15}$
\item $-\dfrac{19}{2}$
\item $\dfrac{2157}{625}$
\item $-\dfrac{17}{8}$
\item $\dfrac{327}{500}$
\end{enumerate}
\end{multicols}
\subsubsection{Show that the decimal expansion of following rational
numbers are non-terminating and
repeating.}
\begin{multicols}{4}
\begin{enumerate}[a.]
 \item $\dfrac{8}{3}$
 \item $\dfrac{2}{11}$
 \item $\dfrac{1}{7}$
 \item $-\dfrac{16}{45}$
\end{enumerate}
\end{multicols}
%=========== End section 1==============%
\section{Section : Non-terminating and recurring rational number}
There are rational numbers such that when we try to express them in
decimal form by
division method, we find that no matter how long we divide there is
always a remainder. In
other words, the division process never comes to an end. This is due to
the reason that in the
division process the remainder starts repeating after a certain number
of steps. In such cases,
a digit of a block of digits repeats itself. For example,
0.3333...,0.1666666...,
0.123123123...., 1.2692307692307692307...etc. Such decimals are called
non-terminating
repeating decimals or non-terminating recurring decimals. These decimal
numbers are
represented by putting a bar over the first block of the repeating part
and omit the other
repeating blocks. Thus, we write 0.33333... = 0.3, 0.16666... =
$\overline{0.16}$, 0.123123123 ..... = $\overline{0.123}$ and
1.26923076923076292307... = $\overline{1.2692307}$.\\\\
Fact : Every non terminating and recurring number is a rational
number.\\\\
1) Express the following rational numbers as decimals
\begin{multicols}{5}
\begin{enumerate}[a.]
 \item $\dfrac{2}{3}$
 \item $-\dfrac{4}{5}$
 \item $-\dfrac{2}{15}$
 \item $-\dfrac{22}{13}$
 \item $\dfrac{437}{999}$
 \item $\dfrac{33}{26}$
\end{enumerate}
\end{multicols}

\noindent 2) Look at several examples of rational numbers in the form
$\dfrac{p}{q} (\neq 0)$, where p and q are integers with no common
factors other than 1 and having terminating decimal representations. Can
you guess what property q must satisfy?
%===================== SECTION 2 END======================%
%\section{Conversion and non-terminating recurring rational numbers in
%the form $\dfrac{p}{q}$
%1) Express 0.\overall{4} in the form $\dfrac{p}{q}$
%2) Express each of the following decimals in the form 
%\begin{multicols}{2}
%\begin{enumerate}[a.]
%\item 0.\overall{35}
%\item 0.\overall{585}
%CORRECT
%\end{enumerate}
%\end{multicols}
%3) Express the following decimals in the form $\dfrac{p}{q}$
%\begin{multicols}{3}
%\begin{enumerate}[a.]
%\item $0.3\overall{2}$
%\item $0.12\overall{3}$
%\item $0.003\overall{52}$
%\end{enumerate}
%\end{multicols}
%3) Express each of the following decimals in the form $\dfrac{p}{q}
%\begin{multicols}{7}
%\begin{enumerate}[a.]
%\item $0.\overall{4}$
%\item $0.\overall{37}$
%\item $0.\overall{54}$
%\item $0.\overall{621}$
%\item $125.\overall{3}$
%\item $4.\overall{7}$
%\item $0.\overall{47}$
%\end{enumerate}
%\end{multicols}
\end{document}
